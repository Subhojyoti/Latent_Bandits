%!TEX root = bandit_paper.tex

\newcommand{\transpose}{^\mathsf{\scriptscriptstyle T}}
\textbf{Notation:} We denote $[n] = \{1, \dots, n\}$ as the set of the first $n$ positive integers. Let $M$ denote any arbitrary matrix of size $m \times n$ matrix. Let the rank of the matrix be denoted by $d$. We denote by $A^B$ the set of all vectors whose entries take values from set $A$ and are indexed by set $B$, where $A$ and $B$ can be any arbitrary sets. For any $d$ and $I \in [m]^d$, $M(I, :)$ denotes a $d \times n$ submatrix of $M$ whose $i$-th row is $M(I(i), :)$. Similarly, for any $d$ and $J \in [n]^d$, $M(:, J)$ denotes a $m \times d$ submatrix of $M$ whose $j$-th column is $M(:, J(j))$. Finally, we denote $\Pi_d$ as the set of all $d$-permutations. For an element $\pi \in \Pi_d$ and $d$-dimensional vector $v$, we denote by $\pi(v)$ the permutation of the entries of $v$ according to $\pi$.
%We index the rows and columns of the matrices by vectors.
%\todob{The previous sentence does not make any sense. What is the set of size $m \times n$?}
%\todoan{We don't index the rows and columns by vectors, but define what it means to use vector indexing. You can just remove this sentence.}

\textbf{Rank-$1$ Setting:} We study the online learning problem of finding the maximum entry of a family of non-stochastic, low-rank and non-negative matrices  which we call as the \emph{non-stochastic low-rank bandit problem}. We first analyze the simple rank-$1$ setup and propose our solution for this setting. Many of the key aspects of our design principle are captured in this rank-$1$ setting. Let $U_t\in [0,1]^{K\times 1}$ be the row preference over a single topic and $V_t \in [0,1]^{L\times 1}$ be the column preference over a single topic. The row-column \emph{preference matrix} $M_t = U_tV_t^{\intercal}$ is non-negative, non-stochastic, and rank $1$. We assume that
\begin{align}
  i^\ast = \argmax_{i \in [K]} U_t(i), \hspace{1.5em}
  j^\ast = \argmax_{j \in [L]} V_t(j) \label{eq:rank1ij}.
\end{align}
for all $t \in [n]$, and some fixed $i^\ast \in [K]$ and $j^\ast \in [L]$. This assumption makes sure that the row and column preferences can change over time $t$, but the best row $i^*$ and column $j^*$ does not change. At every round $t$, the learner chooses one pair of row and column indices $i_t$ and $j_t$, respectively, and observes their product $U_t(i_t)V_t(i_t)$ as its feedback.

\textbf{Regret Definition (Rank-$1$):} The goal of the learner is to minimize the expected $n$-step regret with respect to the optimal solution in the hindsight as follows,
\begin{align}
  R(n) =
  \sum_{t = 1}^n \E\left[M_t(i^\ast, j^\ast) - M_t(i_t,j_t)\right]\,,
  \label{eq:regret}
\end{align}
where the expectation is over any random choice of rows and columns. 

%\todob{Please do not use non-standard notation. $\{\}$ is a set!!! You want to write either $\argmax_{i \in [K], \, t \in [n]} U_t(i, 1)$ or $\argmax_{(i, t) \in [K] \times [n]} U_t(i, 1)$. You also need to move this earlier. You want to say that the best row and column does not change over time.}

%At time $t$, the preferences of users over columns are encoded in a $K \times L$ \emph{preference matrix} $M_t = U_t V_t\transpose$, where $M$, where  the rank of $M$ is $d=1$.
%$U_t$, and $V_t$ are defined as in \cref{sec:background}
%Note that for $d=1$, 

%We study the online learning problem of finding the maximum entry of a non-stochastic, low-rank and non-negative matrix $M$ which we call as a \emph{non-stochastic low-rank bandit problem}. At time $t$, the preferences of users over columns are encoded in a $K \times L$ \emph{preference matrix} $M_t = U_t V_t\transpose$, where $M$, $U_t$, and $V_t$ are defined as in \cref{sec:background}. We assume that row and column preferences ($U_t$ and $V_t$ respectively) can change with time $t$. At every timestep $t$ the learner chooses one pair of row and column indexed by $i_t$ and $j_t$ and observes their product $U_t(i_t)V_t(i_t)$ as its feedback.



%Let $\Pi_d$ be the set of all $d$-permutations. For any $\pi \in \Pi_d$ and $d$-dimensional vector $v$, we denote by $\pi(v)$ the permutation of the entries of $v$ according to $\pi$.

%\todob{This section seems like an exact copy of the arxiv paper. Please change language. Otherwise we may be accused of plagiarism.}

%\textbf{Hott-topics Assumption:} We focus on a family of low-rank matrices, which are known as hott topics. We define a \emph{hott-topics matrix} of rank $d$ as $M = U V\transpose$, where $U$ is a $K \times d$ non-negative matrix and $V$ is a $L \times d$ non-negative matrix that gives rise to the hott-topics structure. In particular, we assume that there exists $d$ rows $I^\ast$ in $U$ such that each row in $U$ can be represented as a convex combination of rows of $I^\ast$ and the zero vector. Hence, for an $A = \{a \in [0, 1]^{d \times 1}: \|a\|_1 \leq 1\}$ each row of $U$ can be expressed as,
%\begin{align}
%  \forall i \in [K] \ \exists \alpha \in A: U(I^\ast, :) \alpha = U(i, :)\,,
%  \label{eq:hott topics1}
%\end{align}
%Similarly, we assume that there exist $d$ rows $J^\ast$ in $V$ such that each row of $V$ can be expressed as a convex combination of rows $J^\ast$ and the zero vector,
%\begin{align}
%  \forall j \in [L] \ \exists \alpha \in A: V(J^\ast, :) \alpha = V(j, :)\,,
%  \label{eq:hott topics}
%\end{align}
%where $A = \{a \in [0, 1]^{d \times 1}: \|a\|_1 \leq 1\}$. Note that we refer to the rows as users and to the columns as columns because this is a standard terminology in recommender systems, where we envision applications of our work. Hence, the matrix $M$ represents preferences of users for columns, $M(i, j)$ is the preference of user $i$ for column $j$. The rank $d$ of $M$ is the number of latent topics. The matrix $U$ are latent preferences of $K$ users over $d$ topics, where $U(i, :)$ are the preferences of user $i \in [K]$. The matrix $V$ are latent preferences of $L$ columns in the space of $d$ topics, where $V(j, :)$ are the coordinates of column $j \in [L]$. Without loss of generality, we assume that $U \in [0, 1]^{K \times d}$ and $V \in [0, 1]^{L \times d}$. We assume that the coordinates are points in a simplex, that is $\|U(i, :)\|_1 \leq 1$ for all $i \in [K]$ and $\|V(j, :)\|_1 \leq 1$ for all $j \in [L]$. Note that our assumptions imply that $M(i, j) \geq 0$ for any $i \in [K]$ and $j \in [L]$.

%\todob{We say that the rows are users and that the columns are columns. But they do not have to be, right? Say that we refer to the rows as users and to the columns as columns because this is a standard terminology in recommender systems, where we envision applications of our work.}
