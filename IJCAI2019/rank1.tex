%!TEX root = bandit_paper.tex

%We first analyze the simple rank $1$ scenario and propose our solution for this setting. Many of the key aspects of our design principle are captured in this rank-$1$ setting. Note that for $d=1$, $U_t\in [0,1]^{K\times 1}$ is the user preference over a single topic and $V_t \in [0,1]^{L\times 1}$ is the item preference over a single topic. The user-item preference matrix $M_t = U_tV_t^{\intercal}$ is non-negative, non-stochastic, and rank $1$. 


%Also, the hott-topics assumption in eq \eqref{eq:hott topics1} and \eqref{eq:hott topics} still holds in the rank $1$ setting.  In this setting, at every timestep $t\in[n]$ the learner selects a pair of rows and columns, denoted by $i_t$ and $j_t$ respectively and observes the feedback $M_t(i_t,j_t)$.
%
%\todob{Explain why.}
%\todob{Use \textbackslash eqref when referring to equations.}

We present the $\latentranker$ for the rank $1$ setting in Algorithm \ref{alg:LRB}. The $\latentranker$ consist of two key components, a row learning algorithm and a column learning algorithm. At every round $t,$ the row algorithm suggests the row $i_t \in [K]$ and the column algorithm suggests the column $j_t\in [L]$. Note that in this non-stochastic scenario we use instances of $\expthree$ as the row and column learning algorithm. The row $\expthree$ has $K$ arms and the column $\expthree$ has $L$ arms. The main idea is to use the row $\expthree$ to learn the best row on average while the column $\expthree$ learns the best column on average. The learner then observes the reward $M_t(i_t, j_t)$ and updates the row and column $\expthree$ simultaneously. A key insight to this design is that when both the row and column learners are run simultaneously, they learn the most rewarding row and column on average and converge on the maximum entry of the matrix $M$. From the definition of $U_t$, for any sequence of $n$ columns, the maximum value is in row $i^\ast$ (see eq \eqref{eq:rank1ij}). This means that the row algorithm learns irrespective of what the column algorithm does. From the definition of $V_t$, for any sequence of $n$ rows, the maximum value is in column $j^\ast$ (see eq \eqref{eq:rank1ij}). This means that the column algorithm learns irrespective of what the row algorithm does. %Although other entries may change arbitrarily in $M_t$, $i^*$ and $j^*$ remain fixed for all $t\in [n]$.
%\todob{Again, this property of $U_t$ and $V_t$ needs to stated formally. They cannot be arbitrary, as you claimed earlier.}
%\todob{This property of $U_t$ and $V_t$ needs to stated formally. They cannot be arbitrary, as you claimed earlier.}
%\todob{Confusing. A name followed by its abbreviation.} in Algorithm \ref{alg:LRB} 
%simple \todob{Avoid repetitive use of ``simple''. You make it sound like we do only easy things.}
%\todob{Use ``round'' instead of time, timestep, step, and such.}
%\todob{We need to clearly explain what is going on here. From the definition of $U_t$, for any sequence of $n$ columns, the maximum value is in row $i^\ast$. This means that the row algorithm learns no matter what the column algorithm does. From the definition of $V_t$, for any sequence of $n$ rows, the maximum value is in column $j^\ast$. This means that the column algorithm learns no matter what the row algorithm does.}
\begin{algorithm}[t]
  \caption{Low Rank Bandit ($\latentranker$) (Rank-$1$)}
  \label{alg:LRB}
  \begin{algorithmic}[1]
    \State \textbf{Input:} Time horizon $n$
    \Comment{Initialization}
      \State Initialize $\rowalg $
      \State Initialize $\colalg $
    \For{$t = 1, \dots, n$}
      \Comment{Generate response}
        \State Row $i_t$ suggested by $\rowalg$
        \State Column $j_t$ suggested by $\colalg$
        \State Observe $M_t(i_t, j_t)$
      \Comment{Update statistics}
    \State Update arm $i_t$ of $\rowalg$ with reward $M_t(i_t, j_t)$.
    \State Update arm $j_t$ of $\colalg$ with reward $M_t(i_t, j_t)$.
     \EndFor
  \end{algorithmic}
\end{algorithm}
