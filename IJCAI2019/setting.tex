%!TEX root = paper.tex

We study an online learning to rank problem, which we call a \emph{latent ranked bandit}. At time $t$, the preferences of users are encoded in a $K \times L$ \emph{preference matrix} $M_t = U_t V\transpose$, where $M$, $U_t$, and $V$ are defined as in \cref{sec:background}. We assume that user preferences $U_t$ can change with time $t$. A random user $i_t \in [K]$ arrives to the recommender system at time $t$ and we recommend $d$ items $J_t$ to this user. The \emph{reward} for recommending these items is $r_t(i_t, J_t)$, where
\begin{align}
  r_t(i, J) =
  \max \, \{\mu(k) \, M_t(i, J(k)): k \in [d]\}
  \label{eq:reward}
\end{align}
is the reward for recommending items $J$ to user $i$ at time $t$, $J(k)$ is the $k$-th item in $J$, and $\mu(k)$ is the weight of position $k \in [d]$. We assume that higher-ranked positions are more rewarding, $1 \geq \mu(1) \geq \dots \geq \mu(d) \geq 0$. The learning agent \emph{observes} the individual rewards of all recommended items, $M_t(i_t, J_t(k))$ for all $k \in [d]$.

%\todob{We need to motivate \eqref{eq:reward} from the user-modeling point of view. This should be the same motivation as in ranked bandits, except that $\mu$ enforces personalization, in the sense that the order matters.} \todoan{See the above comment. To motivate the fractional reward, how about saying that it can correspond to the length of the video the user watches. Like, we recommend a movie/video and if the user watches only half of the video, then the reward is 0.5.}

Since $U_t$ can change arbitrarily over time, the reward in \eqref{eq:reward} is maximized by lists $J$ with highly rewarding items that are diverse, in the sense that they attain high rewards at different times $t \in [n]$. Because the rewards are weighted by $\mu$, more frequent highly-rewarding items should be placed at higher positions. A remarkable property of our user-item preference matrices $M_t$ is that for any user $i \in [K]$ at any time $t$,
\begin{align*}
  \argmax_{j \in [L]} M_t(i, j) \in J_\ast\,,
\end{align*}
where $J_\ast$ is defined in \eqref{eq:hott topics}. Therefore, it is possible to learn all potentially most rewarding items statistically efficiently.

Now we are ready to define our notion of optimality and regret. Let $J_\ast$ be the hott-topics items in \eqref{eq:hott topics} and $\pi_{\ast, i}$ be their permutation that maximizes the reward of user $i$ in hindsight,
\begin{align*}
  \pi_{\ast, i} =
  \argmax_{\pi \in \Pi_d} \sum_{t = 1}^n r_t(i, \pi(J_\ast))\,.
\end{align*}
Let $J_t$ be our recommended items at time $t$ and $\pi_{t, i}$ be their permutation for user $i$, both of which are learned. Then our goal is to minimize the expected $n$-step regret,
\begin{align}
  R(n) =
  \sum_{t = 1}^n \E\left[r_t(i_t, \pi_{\ast, i_t}(J_\ast)) - r_t(i_t, \pi_{t, i_t}(J_t))\right]\,,
  \label{eq:regret}
\end{align}
where the expectation is with respect to both randomly arriving users and potential randomness in the learning algorithm.

%\todob{We do not really need the greedy definition of $J_\ast$ until the proof.}
