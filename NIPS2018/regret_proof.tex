\newcommand\myworries[1]{\textcolor{red}{#1}}

\section{Analysis}
\label{sec:analysis}

We assume that users come sequentially $i_1, \dots, i_n \in [K]$. We denote by $j^*(i)$ the optimal arm of user $i$. When $J =(j_1,...,j_k) \in [L]^k$ is a $k$-tuple, by $J[l]$ we will mean $j_l,$ the $l$'th entry of $J$ and  $\max R(i, J) := \max_{l \in [k]} R(i,J[l])$. \todob{``l'' is a horrible letter because it looks like many other symbols. What about $\ell$ instead?} Let $U_t \in \mathbb{R}_{\geq 0}^{K \times d}$ and $V_t \in \mathbb{R}_{\geq 0}^{L \times d}$ to be time varying latent user and item factors.
 The reward matrix at time step $t \in [n]$ is $R_t = U_t V_t^T.$
\begin{assumption}[Hott Topics]
We will assume that there is a $d$-tuple $J^{*} \in [L]^d$ such that for every  $j \in [L],$ there exists $\alpha_1^j,...,\alpha_d^j \geq 0, \sum_k \alpha_k^j \leq 1$ and
$$ V_t[j,:] = \sum_{k \in J^*} \alpha_k^j V_t[k,:], $$
for every $t \in [n]$.
\end{assumption}
 An important thing to note is that $\alpha_k^j$'s are {\it independent} of time $t.$
With the above assumption, we have the following theorem.  

\begin{lemma}
\label{thm:hott}
For any set of columns $J_t$, we have 
$$ \max_{j \in [L]} \sum_{t \in [n]} \max R_t(i_t, (J_t,j)) = \max_{l \in [d]} \sum_{t \in [n]} \max R_t(i_t, (J_t,J^*[l])).$$
\end{lemma}
\myworries{Todo: What happens with the Bernoulli rounding trick? The above holds when the input is non-stochastic, but I haven't thought about the stochastic case.}
Without loss of generality, we will also assume that for every $1 \leq k \leq d,$ $\max_{J \in [L]^k} \sum_t \max R_t(i_t,J) = \sum_t \max R_t(i_t,J^*[1:k])$ \todoan{Need to say why this is possible: follows easily from hott topics}.  Let $J_t = (\tilde{j}_{t,1},\tilde{j}_{t,2},...,\tilde{j}_{t,d})$ be the tuple of $d$ columns chosen by column-bandits at time $t$ and $(j_{t,1},...,j_{t,d})$ be the $d$ tuple of columns chosen by $i_t$th row EXP3 at time $t.$ We want to bound the expected regret $R(n)$  
$$R(n) = \E \left( d\sum_{t } R(i_t,j^*_t(i_t)) - \sum_{t} \sum_k R(i_t,j_{t,k}) \right).$$
 The row algorithm plays either one arm in $J_t$ $d$ times or plays every arm one time. We will use an indicator function $\mathbbm{1}(j_{t,1} \neq j_{t,2})$ which takes value one only if the row algorithm \todoan{row algorithm? We need a better way to refer to the various EXP3s} plays every arm in $J_t$ one time. Let $p_t = P(j_{t, 1} \neq j_{t, 2})$. We can write the expected regret as 
$R(n) = R_c(n) + R_r(n),$ where
\begin{align*}
  R_c(n) = \E \left( d\sum_{t = 1}^{n} R(i_t,j^*_t(i_t)) - d\sum_{t} \frac{\max R(i_t,J_t)}{p_t} \mathbbm{1}(j_{t,1} \neq j_{t,2})  \right)
\end{align*}
and
\begin{align*}
  R_r(n) = \E \left( d \sum_{t} \frac{ \max R(i_t,J_t)}{p_t} \mathbbm{1}(j_{t,1} \neq j_{t,2}) - \sum_{t} \sum_k R(i_t,j_{t,k}) \right)\,.
\end{align*}
%\todob{The factor of $d$ should only appear at the first term, right?}
  
%Note that 
%$$ \bigcup_k \{ j_{t,k} \} = \{ J^*(i_t)\}$$
We will show that for every $\gamma > 0,$   
$R_c(n) =  O \left( \frac{d^2}{\gamma} \sqrt {n L \log n} \right)$
and $R_r(n) =O \left( \frac{K d\log d}{\gamma} + \gamma n  \right) .$ 
%For sufficiently big $n$, by optimizing over $\gamma$, we get
%Optimizing over $\gamma$
%$$R(n) = O\left( d^{1/2} L^{1/4} n^{3/4}  \log^{1/4} n \right).$$
\begin{theorem}
By choosing $\gamma$ appropriately, for all large enough $n$, we have 
$$R(n) = O\left( d L^{1/4} n^{3/4} \log^{1/4} n\right) .$$
\end{theorem}
We now prove the bounds for $R_c(n)$ and $R_r(n)$ separately.

\myworries{Todo: Make it clear what the randomness is when using $\E$ throughout.}

\todob{Given the limited time, let's go with the current setting. This is most natural in the non-stochastic community and nobody will question it. Then the only randomness is with respect to random actions of the algorithm.}


\subsection{Bounding Column Regret}

To bound $R_c(n)$, we first rewrite it as 
\begin{align*}
R_c(n) &=  d \E \left( \sum_{t = 1}^{n} \frac{R(i_t,j^*_t(i_t))}{p_t} \mathbbm{1}(j_{t,1} \neq j_{t,2}) - \sum_{t} \frac{\max R(i_t,J_t)}{p_t} \mathbbm{1}(j_{t,1} \neq j_{t,2})  \right) \\
&=  d \E \left( \sum_{t = 1}^{n} \frac{\tilde{R}(i_t,j^*_t(i_t))}{p_t}  - \sum_{t} \frac{\max \tilde{R}(i_t,J_t)}{p_t}   \right) \\
&\leq  \frac{d}{\min_t p_t} \E \left( \sum_{t = 1}^{n} \max \tilde{R}(i_t,J^*)  - \sum_{t} \max \tilde{R}(i_t,J_t)   \right).
%&\leq  \frac{d}{\min_t p_t}  \left( OPT - \E\sum_{t} \max \tilde{R}(i_t,J_t)   \right).
\end{align*}
Here, we define $ \tilde{R_t}(i,j) = R_t(i,j)\mathbbm{1}(j_{t,1} \neq j_{t,2})$ .
%The hott topics assumption we need to impose is the following. We can have $U_t$ and $V_t$ to be time varying latent user and item factors. But for every $t$, we need $V_t[j,:]$ to be a fixed convex combination of optimal arms 
%$$ V_t[j,:] = \sum_{k \in J^*} \alpha_k V_t[k,:]. $$
%The key thing to note in the above is that $\alpha_i$'s are {\it independent} of time.
%
%With the above assumption, we have the following theorem.
%
%\begin{theorem}
%\label{thm:hott}
%For any set of (time varying) columns $J_t$ of size $|J_t|=l$ for some $l$, 
%$$ \max_{j \in [L]} \sum_t \max R_t(i_t, (J_t,j)) = \max_{j \in J^*} \sum_t \max R_t(i_t, (J_t,j)).$$
%\end{theorem}
%
%Now we will prove that chaining of EXP3 achieves near optimal regret. For simplicity, we will assume $d=2$. We have two EXP3s in this case. We want to show 
%$$ E \sum_t R_t(i_t,J_t) \geq  E \sum_t \max R_t(i_t,J^*) - O(\sqrt{nL}).$$
%Let us define $OPT_1 = \max_j \sum_t R_t(i_t,j) $ and $OPT_2 = \max_{j_1,j_2} \sum_t \max R_t(i_t,(j_1,j_2)) - OPT_1 .$ We have $OPT_1 + OPT_2 = \sum_t \max R_t(i_t,J^*).$ Let us assume $j_1^*$ be the index that achieves $OPT_1$ and $J^* = (j_1^*,j_2^*)$ be the set that achieves $OPT_2.$
%We will now show that gain of the first EXP3 is close to $OPT_1$ and gain of the second EXP3 is close to $OPT_2$. The gain of the first EXP3 is $E \sum_t R_t(i_t,J_t[1]) .$ From EXP3 guarantee, it follows that $E \sum_t R_t(i_t,J_t[1]) \geq OPT_1 - O(\sqrt{nL}).$ 
%Now, let us consider the gain of the second EXP3. We have 
We are now ready to bound the regret. %For simplicity, let us assume $d=2.$ 
To avoid carrying tildes, we denote $\tilde{R}_t$ by $R_t$ in the rest of the proof.
\begin{lemma}
For any $k \in [d]$,
$$ \sum_t \E \max R_t(i_t,J_t[1:k]) \geq  \E \sum_t \max R_t(i_t,J^*[1:k]) - O(k\sqrt{nL}).$$ 
\end{lemma}
\begin{proof}
We will show this by induction. Note that there are $d$ column EXP3s in this case. The base case when $k=1$ follows because of the guarantees of the first col-EXP3.
Let $J^* = (j_1^*,j_2^*,...,j_d^*).$   We will now assume that the result is true for $k-1$ for some $k>1.$
We have 
\begin{align}
& \E  \sum_t  \max R_t(i_t,J_t[1:k])   \\
&\geq \max_{j_k} \E \sum_t   \max R_t(i_t,(J_t[1:k-1],j_k))  - O\left(\sqrt{nL}\right) \\
&\geq \max_{j_k} \E \sum_t   \max R_t(i_t,(J^*[1:k-1],j_k))   - O\left( \sqrt{nL} \right) - O\left( (k-1)\sqrt{nL} \right)   \\
\label{eq:result}
& =  \E \sum_t   R_t(i_t,J^*[1:k])  - O\left( k\sqrt{nL} \right).
\end{align}
%\todob{The last term in the first line misses $\max$. But a more important question is why do we need that term at all. The term is constant throughout the derivation.}

The last equality follows from Lemma~\ref{thm:hott}. %\todob{Which Theorem 1? Definitely not the one in our paper.} 
The first inequality is from the guarantees of $k$th col-EXP3. \todob{State the regret bound of Exp3 in a lemma.} The crucial step is  the second inequality. It says that we can replace $J_t[1:k-1]$ with $J^*[1:k-1]$ by just losing another additive $O\left( (k-1) \sqrt{nL} \right)$ term. This follows from induction hypothesis and Lemma~\ref{lem:keyinequality}. We note that from Equation~\ref{eq:result}, we have
$$ \max R_t(i_t,J_t[1:k]) \geq  \E \sum_t \max R_t(i_t,J^*[1:k]) - O\left( k \sqrt{nL} \right),$$
which concludes the proof.
\end{proof}
%Define, for  any fixed $j \in [L],$ 
%$$f_2(j_t) = E \sum_t \max R_t(i_t,(j_t,j) .$$
%We have for any $j_1,j_1',j_2$
%$$ \max R_t(i_t,(j_1,j_2))  - \max R_t(i_t,(j_1',j_2)) \geq R_t(i_t,j_1) - R_t(i_t,j'_1) .$$
\begin{lemma}
\label{lem:keyinequality}
Suppose
 $$ \E \sum_t (\max R_t(i_t,(J_t[1:k-1]) \geq  \E \sum_t (\max R_t(i_t,(J^*[1:k-1])  - C$$
 %\todob{Define $C$. Why not just $C$?}
  and let $j_k \in [L].$ Then,
$$ \E \sum_t (\max R_t(i_t,(J_t[1:k-1],j_k) \geq \E \sum_t (\max R_t(i_t,(J^*[1:k-1],j_k) - O\left((k-1) \sqrt{nL} \right).$$
%\myworries{Both Brano and I have confused ourselves over variants of this lemma. Definitely helpful for everyone to verify if the proof is correct.}
\end{lemma}

\begin{proof}
 Let $T_1 = \{ t \, | \max R_t(i_t,J^*[1:k-1]) < R_t(i_t,j_k) \}$ and $T_2 = [n] \backslash T_1 .$ We then have 
\begin{align*}
  & \E \sum_t \max R_t(i_t,(J_t[1:k-1],j_k)) \\
  & \quad = \E\sum_{t \in T_1} \max R_t(i_t,(J_t[1:k-1],j_k)) + \E\sum_{t \in T_2} \max R_t(i_t,(J_t[1:k-1],j_k)) \\
  & \quad \geq \sum_{t \in T_1} \max R_t(i_t,(J^*[1:k-1],j_k)) + \E\sum_{t \in T_2} \max R_t(i_t,(J_t[1:k-1],j_k)) \\
  & \quad \geq \sum_{t \in T_1} \max R_t(i_t,(J^*[1:k-1],j_k)) + \E\sum_{t \in T_2} \max R_t(i_t,J_t[1:k-1]) \\
  & \quad \geq \sum_{t \in T_1} \max R_t(i_t,(J^*[1:k-1],j_k)) + \sum_{t \in T_2} \max R_t(i_t,J^*[1:k-1]) - C \\
  & \quad = \sum_{t \in T_1} \max R_t(i_t,(J^*[1:k-1],j_k)) + \sum_{t \in T_2} \max R_t(i_t,(J^*[1:k-1],j_2)) - C \\
  & \quad = \sum_{t \in [n]} \max R_t(i_t,(J^*[1:k-1],j_k)) - C.
\end{align*}
%\todob{$(j_1^*,j_2)$ should be $(J^*[1:k-1],j_k)$. $j_2$ should be $j_k$. The same below.}

The first inequality is easy because $\max R_t(i_t,(J^*[1:k-1],j_k) = R_t(i_t,j_k)$ for $t \in T_1$. Second inequality is trivial. Third inequality follows from the assumption. The next equality holds because of the definition of $T_2$.  
\end{proof}
\todoan{Define a new slicing operator and a more compressed '-' operator so that the above expressions look a bit nicer?}
%Therefore, 
%\begin{align*}
%E \sum_t \max R_t(i_t,(J_1[t],j))  - & E \sum_t \max R_t(i_t,(j_1^*,j)) \\
%& \geq E \sum_t R_t(i_t,J_1[t]) - R_t(i_t,j_1^*) \\
%&\geq -O(\sqrt{nL}),
%\end{align*}
%where the last inequality follows from first EXP3 guarantee.


\subsection{Bounding Row Regret}

To bound $R_r(n)$, we first note that
$$R_r(n) =  E \left( \sum_{t} d\cdot \max R(i_t,J_t)  - \sum_{t} \sum_k R(i_t,j_{t,k}) \right) .$$ 
We will decompose the regret as a sum of regret of row-EXP3s. There are $K$ row-EXP3s and each one corresponds to a user. Let $n_i$ be the number of times user $i$ appears in the sequence $i_1,...,i_n$. We then have 
$$R_{r,i}(n) = \sum_{i \in [K]} R_{r,i}(n) $$ 
where $R_{r,i}(n) = E \left( \sum_{t} d \cdot \max R(i,J_t)  - \sum_{t} \sum_k R(i,j_{t,k}) \right) .$ Since each user has a row-EXP3 is over $d+1$ arms, the regret is bounded by
$$R_{r,i}(n) =  (e-1)\frac{(d+1)\log (d+1)}{\gamma} + \gamma n_i,$$
where $\gamma >0$ is any positive number. Summing this over $K$ users, we get
$$R(n) =  (e-1)\frac{(d+1)\log (d+1)K}{\gamma} + \gamma n.$$
%Optimizing over $\gamma$, we get $R(n) = O\left( K^{1/2} d^{1/2} \log^{1/2}(d) n^{1/2}  \right) . $
%Since there is an EXP3 algorithm on $d+1$ arms for each user, we have
%$$R_r(n) = \sum_{i \in [K]}O(\sqrt{n_i d} f(\gamma)),$$
%$$R_r(n) = O \left( \frac{K d\log d}{\gamma} + \gamma n  \right),$$
%where $n_i$ is the number of times user $i\in [K]$ was observed, and

\todoan{This proof needs to have a bit more details. Also, $\gamma$ should appear in the algorithm and we should refer to that.}
%\end{document}

\todob{Please add more details. This needs to be done over all users.}
