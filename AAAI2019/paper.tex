\def\year{2019}\relax
%File: formatting-instruction.tex
\documentclass[letterpaper]{article} %DO NOT CHANGE THIS
\usepackage{aaai19}  %Required
\usepackage{times}  %Required
\usepackage{helvet}  %Required
\usepackage{courier}  %Required
\usepackage{url}  %Required
\usepackage{graphicx}  %Required


%%%%%%%%%%%%% ADD CUSTOM PACKAGES TO macros.sty
\usepackage{macros}

%%%%%%%%%%%%%For adding comments

\newcommand{\todob}[2][]{\todo[color=cyan!20,size=\tiny,inline,#1]{B: #2}} % Brano's comments
\newcommand{\todosb}[2][]{\todo[color=green!20,size=\tiny,inline, #1]{S: #2}} %
\newcommand{\todoan}[2][]{\todo[color=black!20,size=\tiny,inline, #1]{A: #2}} %

\frenchspacing  %Required
\setlength{\pdfpagewidth}{8.5in}  %Required
\setlength{\pdfpageheight}{11in}  %Required
%PDF Info Is Required:
  \pdfinfo{
/Title (2019 Formatting Instructions for Authors Using LaTeX)
/Author (AAAI Press Staff)}
\setcounter{secnumdepth}{2}  
 \begin{document}
% The file aaai.sty is the style file for AAAI Press 
% proceedings, working notes, and technical reports.
%
\title{Latent Ranked Bandit}
\author{Author names withheld}
%\author{AAAI Press\\
%Association for the Advancement of Artificial Intelligence\\
%2275 East Bayshore Road, Suite 160\\
%Palo Alto, California 94303\\
%}
\maketitle
\begin{abstract}
We study the problem of learning personalized ranked lists of diverse items for multiple users, from sequential observations of user preferences. The user-item preference matrix is non-negative and low-rank. Existing methods for solving similar problems are based on reconstructing the preference matrix from its noisy observations using matrix factorization techniques, and typically require strong assumptions on the reconstructed matrix. We depart from this standard approach and focus on a family of low-rank matrices, where the set of most preferred items of all users is small and can be learned efficiently. Then we learn to present this set to each user in a personalized manner, in the order of the descending preferences of the user. We propose a computationally-efficient algorithm that implements this procedure, which we call latent ranker (LR), and prove a sublinear bound on its $n$-step regret. We evaluate the algorithm empirically on several synthetic and real-world datasets. In all experiments, we outperform existing state-of-the-art algorithms.
\end{abstract}

\section{Introduction}
\label{intro}
In this work, we study the problem of learning the maximum entry of a low-rank matrix from sequential observations. These type of low-rank structure is observed in many real-world applications and is a standard assumption in recommender systems \citep{koren2009matrix,ricci2011liorrokach}. Our learning model is motivated by a real-world scenario, where a marketer wants to advertise a product and has $K$ population segments and $L$ marketing channels. Now, given a product some population segment prefer some marketing channels more than other. Hence, a successful conversion happens if each population segment is matched to the correct marketing channel which is nothing but the maximum entry of the matrix formed by the outer product of the users preference and marketing channels over some number of common topics.  

We formalize our learning problem as the following online learning problem. At time $t$, the learning agent chooses $d$-pairs of rows and columns where $d$ is the rank of a non-negative and low-rank matrix $M$. We use the terminology row/user and column/item interchangeably, keeping in sync with our proposed application area . This user-item preference matrix $M$ is formed by the outer product of user and item latent preferences over $d$ topics. Note that the learner does not observe the individual latent values of user or item preferences but just their product. The user-item preference matrix $M$ is low-rank at each round $t$, can vary substantially over time, and does not have to be stochastic. The goal of our learning agent is to minimize the cumulative regret with respect to a best solution in hindsight by finding the maximum entry in $M$ as quickly as possible.



Previous works that have studied this setting have either proposed highly conservative algorithms or restricted themselves to a stricter set of assumptions. While \citet{katariya2016stochastic} was proposed for a rank $1$ bandit model with the assumption that the underlying distributions are stochastic, \citet{katariya2017bernoulli} was proposed for the special case when the underlying distributions are Bernoulli. Both these works used different variations of the phase-based UCB-Improved \citep{auer2010ucb} algorithm to construct a confidence interval set over row-column pairs to identify and eliminate sub-optimal rows and columns. These naturally results in algorithms that explore conservatively (for the sake of row and column elimination) and cannot work beyond the stochastic distribution assumption. Finally, \citet{kveton2017stochastic} can be viewed as a generalization of rank-$1$ bandits of \citet{katariya2016stochastic} to a higher rank of $d$. However, this work proposes a phase-based algorithm that calculates the square of the determinant of a $d\times d$ sub-matrix to eliminate sub-optimal rows and columns at the end of phases which is impractical for very large non-negative low-rank matrices. Some other approaches involving non-negative matrix factorization \citet{sen2016contextual} or tensor based methods \citep{gopalan2016low} to reconstruct the matrix have also been proposed. These works require strong assumptions on the structure of the matrix such as all the matrices satisfy a weak statistical Restricted Isometric Property (RIP) or calculate third order tensors as in \citet{anandkumar2014tensor}. A more simpler setting has also been studied in \citet{maillard2014latent}.

%are detailed in Section \ref{sec:related work}
 %\citet{sen2016contextual} is an online matrix completion algorithm which is an $\epsilon$-greedy algorithm that tries to reconstruct the matrix $M$ through non-negative matrix factorization. Note, that this approach requires that all the matrices satisfy a weak statistical Restricted Isometric Property, which is not always feasible in real life applications. Another approach is that of \citet{gopalan2016low} where the authors come up with an algorithm which uses the Robust Tensor Power (RTP) method of 
%\citet{anandkumar2014tensor} to reconstruct the matrix $M$, and then use the OFUL procedure of \citet{abbasi2011improved} to behave greedily over the reconstructed matrix. 
%But the RTP is a costly operation because the learner needs to construct a matrix of order $L\times L$ and $L\times L \times L$ to calculate the second and third order tensors for the reconstruction.  A more simpler setting has also been studied in \citet{maillard2014latent}

Our approach is based on two key insights. First, the earlier methods (like Upper Confidence Bound (UCB) algorithms, NMF-Bandits \citep{sen2016contextual}) are explicitly modeled on the stochastic i.i.d assumption on feedback and cannot perform well in non-stochastic settings. Moreover, their theoretical guarantees will also fail in non-stochastic setting. Hence, we need algorithms that can work on more generalized non-stochastic probability distribution settings. Secondly, we can formulate simple and computationally efficient algorithms that learn the best set of columns and best set of rows jointly with two separate non-stochastic bandit algorithm operating on rows and columns individually. These does not require any sort of costly matrix inversion or reconstruction operations or even row or column eliminations and hence are faster in implementation. 

%We formalize our learning problem as the following online learning problem. At time $t$, a random user $i_t$ from a pool of $K$ users arrives to the recommender system. The learning agent observes the identity of the user $i_t$, recommends a list of $d$ diverse items $J_t$ from a pool of $L$ items as a response, and observes the preferences of user $i_t$ for all recommended items $J_t$. The user-item preference matrix is low-rank at each time $t$, can vary substantially over time, and does not have to be stochastic. The reward of the recommended list is high when highly preferred items of the user are recommended at higher positions. The goal of our learning agent is to compete with the most rewarding diverse list for each user in hindsight.

%Our learning model is motivated by a real-world scenario, where the learning agent suggests movies to users and each movie belongs to different movie genres. The agent typically does not observe instantaneous preferences of the user, and therefore suggests multiple movies that may be of interest to the user under different circumstances. A similar model has also been studied in \citet{carbonell1998use} where the goal is to suggest a diversified list to each incoming user that combines relevance to the query as well as novelty. The authors suggest an approach where each item in the list is relevant to the query but also has \textit{"marginal relevance"} or less similarity with previously selected documents and this improves the quality of recommendation.

%The key structure of our matrix is that the set of optimal items of all users is small and can be learned jointly for all users. Given these items, the problem of learning the optimal order for each user can be solved in the full-information setting and thus is easy. 

%The column learning algorithm is similar to ranked bandits. In particular, we learn the $k$-th most diverse item using a multi-armed bandit, whose rewards are conditioned on the rewards of $k - 1$ previously chosen items. The row learning problem is solved separately for each user. Because it is in the full-information setting, as we observe the individual rewards of all recommended items,  we solve it using the weighted majority algorithm.

We make four major contributions. First, we formulate our online learning problem as a non-stochastic bandit problem on a class of non-negative low-rank matrices. We identify a family of non-negative low-rank matrices where our problem can be solved statistically efficiently, without actually observing the latent values of individual rows and columns. Second, we propose a computationally-efficient algorithm that implements this idea, which we call Low Rank Bandit ($\latentranker$) algorithm. The algorithm has two components, column learning and row learning, which learn the pair  of optimal columns and rows respectively. Since we are in the non-stochastic setting we use a variation of the $\expthree$ \citep{auer2002nonstochastic} algorithm as our row and column learner. Note, that we do not construct any confidence interval or eliminate rows and columns like the existing works. Infact, we use the well known fact that exponentially weighted algorithm like $\expthree$ are robust and fast learner to design our algorithm. The Third, we analyze $\latentranker$ and up to problem-specific factors, we prove a $O\left(\frac{\left(\sqrt{L } + \sqrt{K }\right)\sqrt{n}}{\Delta}\right)$ upper bound on its $n$-step regret in the special case when rank is $1$. The regret of a naive solution is $O(\sqrt{K L n})$, and is much worse than that of $\latentranker$ when all of $K$, $L$, and $n$ are large. Finally, we evaluate $\latentranker$ empirically on several synthetic and real-world problems. Perhaps surprisingly, $\latentranker$ performs well even when our modeling assumptions are violated.

The paper is organized as follows. We introduce necessary background and settings to understand our work for the rank-$1$ scenario in \cref{sec:background}. We then introduce the rank-$1$ algorithm in \cref{sec:rank1} and bound the regret of this rank-$1$ algorithm in \cref{sec:analysis}. We then analyze the general rank $d$ setting in \cref{sec:setting} and propose our algorithm for the general rank $d$ in \cref{sec:algorithm} . In \cref{sec:experiments}, we evaluate the algorithm empirically. We conclude in \cref{sec:conclusions}. 

%and define our online learning problem in \cref{sec:setting}
%The detailed proof of our regret bound is presented in \cref{sec:proof}.
% In \cref{sec:related work}, we survey related work.

\subsection{Contributions}
\label{Contribution}
Our contributions are mainly three fold. First, we formulate the latent ranked bandit problem as an online learning problem on a class of low-rank non-negative matrices that can be solved efficiently without estimating the latent factors that generate the matrices. We borrow ideas from the ranking literature to solve this problem efficiently and term this setting as personalized ranking setting. This is because an efficient algorithm can find a global ranking of best items and then permute that list to find a personalized ranking for individual users, sorted in descending order of their preference towards item. Secondly, we propose the latent ranker algorithm (LRA) for this personalized ranking setting. LRA has two components, the column algorithms that suggests $d$ columns and the row ranking component which permutes that suggested list to find the best permutation amongst them. The best permutation is personalized for each user such that it will contain the best item for that user at rank $1$ position with high probability. The column algorithms leverage the low rank structure of the user-preference matrix $M$ and quickly finds out the $d$ best items. Simultaneously the row ranking components observe all the feedback from the user permute the suggested list of $d$ items to find the highest reward permutation such that the best item is at rank $1$ position for the individual users. Finally, we show that an instance of LRA which uses exponential weighting algorithm EXP3 as column MABs and weighted majority algorithm (WMA) as row MABs suffer a regret of atmost $O\left(d^2\sqrt{L n} + K \log n\right)$. On diverse experimental settings we test our proposed algorithm and show improved performance even when our modeling assumptions does not hold.

	The rest of the paper is organized as follows. We state our setting, assumptions and notations in Section \ref{sec:setting}. We propose the algorithm LRA in Section \ref{sec:algorithm} and in Section \ref{sec:analysis} we analyze LRA. Finally, we show experiments in Section \ref{sec:expt} and give a brief survey of the existing literature in Section \ref{sec:related}. We conclude in Section \ref{sec:conclusions} while the proof is contained in the Appendix \ref{sec:proof}.

\section{Setting}
\label{sec:setting}
%!TEX root = bandit_paper.tex

In this section, we study the online learning problem of finding the maximum entry of a non-stochastic, low-rank and non-negative matrix for the general rank-$d$ setting. 

\textbf{Hott-topics Assumption:} We focus on a family of low-rank matrices, which are known as hott topics. We define a \emph{hott-topics matrix} of rank $d$ as $M = U V\transpose$, where $U$ is a $K \times d$ non-negative matrix and $V$ is a $L \times d$ non-negative matrix that gives rise to the hott-topics structure. In particular, we assume that there exists $d$ rows $I^\ast$ in $U$ such that each row in $U$ can be represented as a convex combination of rows of $I^\ast$ and the zero vector. Hence, for an $A = \{a \in [0, 1]^{d \times 1}: \|a\|_1 \leq 1\}$ each row of $U$ can be expressed as,
\begin{align}
  \forall i \in [K] \ \exists \alpha \in A: U(I^\ast, :) \alpha = U(i, :)\,,
  \label{eq:hott topics1}
\end{align}
Similarly, we assume that there exist $d$ rows $J^\ast$ in $V$ such that each row of $V$ can be expressed as a convex combination of rows $J^\ast$ and the zero vector,
\begin{align}
  \forall j \in [L] \ \exists \alpha \in A: V(J^\ast, :) \alpha = V(j, :)\,,
  \label{eq:hott topics}
\end{align}
where $A = \{a \in [0, 1]^{d \times 1}: \|a\|_1 \leq 1\}$. Hence, the matrix $M$ represents preferences of users for items, $M(i, j)$ is the preference of user $i$ for item $j$. The rank $d$ of $M$ is the number of latent topics. The matrix $U$ are latent preferences of $K$ users over $d$ topics, where $U(i, :)$ are the preferences of user $i \in [K]$. The matrix $V$ are latent preferences of $L$ items in the space of $d$ topics, where $V(j, :)$ are the coordinates of item $j \in [L]$. Without loss of generality, we assume that $U \in [0, 1]^{K \times d}$ and $V \in [0, 1]^{L \times d}$. We assume that the coordinates are points in a simplex, that is $\|U(i, :)\|_1 \leq 1$ for all $i \in [K]$ and $\|V(j, :)\|_1 \leq 1$ for all $j \in [L]$. Note that our assumptions imply that $M(i, j) \geq 0$ for any $i \in [K]$ and $j \in [L]$.


\textbf{Rank-$d$ Setting:} Again, note that at time $t$, the preferences of users over items are encoded in a $K \times L$ \emph{preference matrix} $M_t = U_t V_t\transpose$, where $U_t$, and $V_t$ are defined as in \eqref{eq:hott topics1} and \eqref{eq:hott topics}. We assume that user and item preferences ($U_t$ and $V_t$ respectively) can change with time $t$. 
At every round $t$ the learner chooses $d$-pairs of rows and columns from $M_t$ denoted by $(I_t,J_t)\in \Pi_d([K])\times \Pi_d([L])$. It then observes all the values from the matrix $M_{t}(I_t,J_t)$ for all $i_t\in I_t$ and $j_t \in J_t$. The \emph{reward} for the agent for choosing arms $(I_t,J_t)$ at time $t$ is denoted by $r_t(i^\ast(I_t,J_t),j^\ast(I_t,J_t))$ such that,
\begin{align}
  (i^\ast(I,J),j^\ast(I,J)) = \argmax_{(i,j) \in (I\times J)} M_{t}(i,j)
  \label{eq:reward}
\end{align}

%a noisy \todob{We do not have any noise, right?} realization of

\todob{Say how the hott topics assumption simplifies the problem of finding the maximum entry of a matrix. Write it formally.}

%We study an online learning to rank problem, which we call a \emph{latent ranked bandit}. At time $t$, the preferences of users are encoded in a $K \times L$ \emph{preference matrix} $M_t = U_t V\transpose$, where $M$, $U_t$, and $V$ are defined as in \cref{sec:background}. We assume that user preferences $U_t$ can change with time $t$. A random user $i_t \in [K]$ arrives to the recommender system at time $t$ and we recommend $d$ items $J_t$ to this user. The \emph{reward} for recommending these items is $r_t(i_t, J_t)$, where
%\begin{align}
%  r_t(i, J) =
%  \max \, \{\mu(k) \, M_t(i, J(k)): k \in [d]\}
%  \label{eq:reward}
%\end{align}
%is the reward for recommending items $J$ to user $i$ at time $t$, $J(k)$ is the $k$-th item in $J$, and $\mu(k)$ is the weight of position $k \in [d]$. We assume that higher-ranked positions are more rewarding, $1 \geq \mu(1) \geq \dots \geq \mu(d) \geq 0$. The learning agent \emph{observes} the individual rewards of all recommended items, $M_t(i_t, J_t(k))$ for all $k \in [d]$.

%\todob{We need to motivate \eqref{eq:reward} from the user-modeling point of view. This should be the same motivation as in ranked bandits, except that $\mu$ enforces personalization, in the sense that the order matters.} \todoan{See the above comment. To motivate the fractional reward, how about saying that it can correspond to the length of the video the user watches. Like, we recommend a movie/video and if the user watches only half of the video, then the reward is 0.5.}

Since $U_t$ can change arbitrarily over time \todob{This obviously cannot be arbitrary. The assumption is that all $M_t$ have the same hott topics rows and columns. Write it formally.}, the reward in \eqref{eq:reward} is maximized by lists $J$ with highly rewarding items that are diverse, in the sense that they attain high rewards at different times $t \in [n]$. A remarkable property of our user-item preference matrices $M_t$ is that for any user $i \in [K]$ and any item $j \in [L]$ at any time $t$,
\begin{align*}
  \argmax_{(i, j) \in ([K] \times [L])} M_t(i, j) \in (I^\ast, J^\ast),
\end{align*}
where $I^\ast$ and $J^\ast$ is defined in \eqref{eq:hott topics1} and \eqref{eq:hott topics}. Therefore, it is possible to learn all potentially most rewarding pairs of rows and columns statistically efficiently.

%\todob{The definition of the regret below is incorrect for $d > 1$. We need to think about this. Actually, do we need the definition of the regret for $d > 1$ if we never bound it? We should use determinants and then intuitively explain what they mean. The rank $1$ case is intuitive and easy to explain.}

\textbf{Regret Definition (Rank-$d$):} Now we are ready to define our notion of optimality and regret for the general rank-$d$ scenario. Our goal is to minimize the expected $n$-step regret,
\begin{align}
  R(n) =
  \sum_{t = 1}^n \E\left[r_t(i^\ast_t, j^\ast_t) - r_t(i^\ast(I,J),j^\ast(I,J))\right]\,,
  \label{eq:regret1}
\end{align}
where the expectation is with respect to both randomly choosing rows $(I_t)$ and columns $(J_t)$ by the learning algorithm and potential randomness in the environment.

%Now we are ready to define our notion of optimality and regret. Let $J_\ast$ be the hott-topics items in \eqref{eq:hott topics} and $\pi_{\ast, i}$ be their permutation that maximizes the reward of user $i$ in hindsight,
%\begin{align*}
%  \pi_{\ast, i} =
%  \argmax_{\pi \in \Pi_d} \sum_{t = 1}^n r_t(i, \pi(J_\ast))\,.
%\end{align*}
%Let $J_t$ be our recommended items at time $t$ and $\pi_{t, i}$ be their permutation for user $i$, both of which are learned. Then our goal is to minimize the expected $n$-step regret,
%\begin{align}
%  R(n) =
%  \sum_{t = 1}^n \E\left[r_t(i_t, \pi_{\ast, i_t}(J_\ast)) - r_t(i_t, \pi_{t, i_t}(J_t))\right]\,,
%  \label{eq:regret}
%\end{align}
%where the expectation is with respect to both randomly arriving users and potential randomness in the learning algorithm.

%\todob{We do not really need the greedy definition of $J_\ast$ until the proof.}


\section{Algorithm}
\label{sec:algorithm}
%!TEX root = bandit_paper.tex

Now, we propose the general \emph{Low Rank Bandit ($\latentranker$)} algorithm for solving the family of non-stochastic, non-negative and low-rank matrices of rank $d$. Again, the goal is to identify the maximum entry of the matrix by quickly identifying the $d$-best rows or columns. \todob{There is no single matrix.} The pseudocode of $\latentranker$ is in \cref{alg:LRB1}. $\latentranker$ has two main components, column learning and row learning algorithm. 

At every timestep \todob{round} $t$, the row learning algorithm recommends a list of $d$ rows and is the same as the Ranked Bandit Algorithm (RBA) in \citet{radlinski2008learning}. \todob{Leave talking about related work for ``Related Work''. It is distracting if you do it here.} But we exploit an additional structure in our problem to show that we learn the optimal rows $I^\ast$. The row learning algorithm are $d$ instances of multi-armed bandit algorithms, which we denote by $\rowalg(k)$ for algorithm $k \in [d]$. $\rowalg(1)$ learns the most rewarding row on average, $\rowalg(2)$ learns the second most rewarding row on average conditioned on the first learned column, and so on.

Similarly, column learning algorithm recommends a list of $d$ columns by exploiting the same structure in our rewards. Again the goal of the column learning algorithm $\colalg$ is to learn the optimal set of columns $J^\ast$. Hence, $\colalg(1), \dots, \colalg(d)$ learns the most rewarding columns $j_1, \dots, j_d$ \todob{This is not the notation for the best columns.} on average. Note, that this sequence of learning the rows or columns first does not matter because the \emph{hott-topics} structure is defined on both $U$ and $V$ matrix generating $M_t = U_t V_t^\intercal$, and so we will be learning the $d$-best rows or columns in average. \todob{This needs to be stated formally. What does it mean mathematically?} Another way of looking at this is to first realize that if we fix the column selection strategy, which is simply some distribution over $d$-tuples of chosen columns then for any such distribution, the $d$ hott-topic rows are the optimal solution to the row selection problem. By symmetry, the same is true for the column selection problem. If we run both in parallel, and the distributions in the other dimensions do not change too fast (this is true by our design), then $i_1, \dots, i_d$ and $j_1, \dots, j_d$ would slowly converge to the $d$ hott-topic rows and columns.


%Here is an informal argument for the correctness of the algorithm. Fix the column selection strategy, which is simply some distribution over d-tuples of chosen columns. For any such distribution, the d hott topic rows are the optimal solution to the row selection problem. By symmetry, the same is true for the column selection problem. If we run both in parallel, and the distributions in the other dimensions do not change too fast (this is true by our design), I would think that i_1, \dots, i_d and j_1, \dots, j_d would slowly converge to the d hott topic rows and columns.


Finally, $\latentranker$ observes the individual rewards of $M(i,j)$ for all $(i,j)\in (I_t, J_t)$. Then we update both column and row learning algorithms. The reward of the arm in $\rowalg(k)$, which selects the $k$-th row in $I_t$, is updated as follows. If the $k$-th arm was not previously suggested row its reward is updated $d$ times such that, $\max_{k \leq k_1} M_t(i_k, j_{k_2}) - \max_{k < k_1} M_t(i_k, j_{k_2})$ for all $k_1,k_2 \in [d]$. By the choice of our design and previous argument a similar update is performed on the $k$-th column learning algorithm $\colalg$ such that its reward is also updated $d$ times such that, $\max_{k \leq k_2} M_t(i_{k_1}, j_k)  - \max_{k < k_2} M_t(i_{k_1}, j_k)$ for all $k_1,k_2 \in [d]$. Otherwise, if any of the row or column has been previously selected by the corresponding $\rowalg$ or $\colalg$ algorithm then we update it with reward $0$. \todob{Explain why this is needed.}


%The reward of the arm in $\colalg(k)$, which selects the $k$-th column in $J_t$, is updated as follows. If the arm was not one of the previously suggested columns, its reward is $\max \, \{M_t(i_t, J_t(a)): a \in [k]\} - \max \, \{M_t(i_t, J_t(a)): a \in [k - 1]\}$. Otherwise, we update the initially suggested arm with reward $0$. Since $\latentranker$ observes the individual rewards of all recommended items, we can compute the reward of any permutation of $J_t$ in row $i_t$. These rewards are then used to update $\rowalg(i_t, J_t)$. 


%It then observes the reward $u_t(i_t)v_t(j_t)$. The $\colalg$ learns the most rewarding column on average, while $\rowalg$ learns the most rewarding row on average. Once it observes the reward it updates the statistics of each algorithm with the reward $u_t(i_t)v_t(j_t)$.

%The column learning algorithm recommends a list of $d$ columns and is the same as in \citet{radlinski2008learning}. But we exploit an additional structure in our problem to show that we learn the optimal columns $J_\ast$. The column learning algorithm are $d$ instances of multi-armed bandit algorithms, which we denote by $\colalg(k)$ for algorithm $k \in [d]$. $\colalg(1)$ learns the most rewarding column on average, $\colalg(2)$ learns the second most rewarding column on average conditioned on the first learned column, and so on.
%
%The row ranking algorithm permutes columns suggested by the column learning algorithm. It consists of multiple instances of full-information algorithms. More precisely, for each user $i \in [K]$ and set of $d$ columns $J$, we have algorithm $\rowalg(i, J)$ with $d!$ arms, which correspond to all possible permutations of $J$. The objective of $\rowalg(i, J)$ is to learn a permutation of $J$ with the highest reward, as measured by \eqref{eq:reward}.
%
%$\latentranker$ interacts with the environment as follows. At time $t$, a random user $i_t$ is revealed to $\latentranker$. Then, in the ascending order of $k \in [d]$, $\colalg(k)$ suggests column $\ell_k$. If $\colalg(k)$ suggests one of the previously suggested columns $\ell_1, \dots, \ell_{k - 1}$, then $\ell_k$ is chosen uniformly at random from the remaining columns. We denote the vector of $d$ suggested columns  by $J_t$. Then $\rowalg(i_t, J_t)$, the row learning algorithm for user $i_t$ and columns $J_t$, selects permutation $\pi_{t, i_t}$ of $J_t$.
%
%The user is recommended a permuted list $\pi_{t, i_t}(J_t)$ and $\latentranker$ observes the individual rewards of all recommended items. Then we update both column and row learning algorithms. The reward of the arm in $\colalg(k)$, which selects the $k$-th column in $J_t$, is updated as follows. If the arm was not one of the previously suggested columns, its reward is $\max \, \{M_t(i_t, J_t(a)): a \in [k]\} - \max \, \{M_t(i_t, J_t(a)): a \in [k - 1]\}$. Otherwise, we update the initially suggested arm with reward $0$. Since $\latentranker$ observes the individual rewards of all recommended items, we can compute the reward of any permutation of $J_t$ in row $i_t$. These rewards are then used to update $\rowalg(i_t, J_t)$. 


\begin{algorithm}[t]
  \caption{Low Rank Bandit ($\latentranker$) (Rank-$d$)}
  \label{alg:LRB1}
  \begin{algorithmic}[1]
    \State \textbf{Input:} Time horizon $n$, Rank $d$
    \For{$k = 1, \dots, d$}
    \Comment{Initialization}
      \State Initialize $\rowalg (k)$
      \State Initialize $\colalg (k)$
    \EndFor
    \For{$t = 1, \dots, n$}
    \For{$k = 1, \dots, d$}
      \Comment{Generate response}
        \State $\hat{i}_k \gets$ Suggested row $i_t$ by $\rowalg(k)$
        %\todob{$\hat{b}$ for rows is very confusing. What about $\hat{i}$? The same goes for $b$.}
        \If{$\hat{i}_k \in \{i_1, \dots, i_{k - 1}\}$}
          \State $i_k \gets$ Random row not in $\{i_1, \dots, i_{k - 1}\}$
        \Else
          \State $i_k \gets \hat{i}_k$
        \EndIf
        \State $\hat{j}_k \gets$ Suggested column $j_t$ by $\colalg(k)$
        %\todob{$\hat{\ell}$ for columns is very confusing. What about $\hat{j}$? The same goes for $\ell$.}
        \If{$\hat{j}_k \in \{j_1, \dots, j_{k - 1}\}$}
          \State $j_k \gets$ Random column not in $\{j_1, \dots, j_{k - 1}\}$
        \Else
          \State $j_k \gets \hat{j}_k$
        \EndIf
      \EndFor
      \State $I_t \gets (i_1, \dots, i_d)$
      \State $J_t \gets (j_1, \dots, j_d)$
      \State Observe $M_t(I_t(k), J_t(k))$ for all $k \in [d]$
      \For{$k_1 = 1, \dots, d$}
      \For{$k_2 = 1, \dots, d$}
      \Comment{Update statistics}
        \If{$i_{k} = \hat{i}_{k}$}
        \State Update arm $i_{k}$ of $\rowalg(k)$ with reward
       \begin{align*}
    &\qquad \qquad \max_{k \leq k_1} M_t(i_k, j_{k_2}) - \max_{k < k_1} M_t(i_k, j_{k_2})
        \end{align*}
        \Else
          \State Update $\hat{i}_k$ of $\rowalg(k)$ with reward $0$ 
        \EndIf
        \If{$j_{k} = \hat{j}_{k}$}
    		\State Update arm $j_{k}$ of $\colalg(k)$ with reward
         \begin{align*}
         &\qquad \qquad \max_{k \leq k_2} M_t(i_{k_1}, j_k)  - \max_{k < k_2} M_t(i_{k_1}, j_k)
        \end{align*}
        \Else
          \State Update $\hat{j}_k$ of $\colalg(k)$ with reward $0$
        \EndIf
      \EndFor
      \EndFor
     \EndFor
  \end{algorithmic}
\end{algorithm}

%\todob{Simplify the presentation of the update below. To start with, break it into row and column components.}

\subsection{Practical Considerations}
\label{sec:practical considerations}

%The proposed $\latentranker$ algorithm only has to update/look through $(Kd + d)$ items for each of the $d$ $\colalg$ and the i-th $\rowalg$ at every timestep $t$. This is in stark contrast to some of the existing matrix completion algorithms which has to reconstruct a $K\times L$ matrix \citep{sen2016contextual} or calculate second or third order tensors \citep{gopalan2016low}. 
%\todob{Discuss time and space complexities of $\latentranker$.}


%\todob{Say how we hack $\rowalg$. 
%We do not have one row Exp3 for each user and any combination of columns, right? %This is how LRA is proposed. This needs to be said.}
%$\latentranker$-rank$2$

We leave the implementation of the $\colalg$ and $\rowalg$ to the users. For theoretical guarantees \todob{What guarantees? We have none. Say that this is motivated by rank $1$.} we use non-stochastic algorithm $\expthree$ as $\colalg$ and $\rowalg$ and showed how the regret scales for the rank-$1$ setting. For experimental purposes, stochastic algorithms like $\ucb$ or Thompson Sampling can also be used to improve the performance of $\latentranker$. This has also been explored in \citet{radlinski2008learning} where $\RBA$ uses $\ucb$ for ranking items. The proposed $\latentranker$ algorithm only has to update/look \todob{I am pretty sure that you can use just one of these words. Why would you say update/look?} through $(K + L)d$ entries for the $d$ $\colalg$ and the $\rowalg$ respectively at every round $t$. This is in stark contrast to some of the existing matrix completion algorithms which has to reconstruct a $K\times L$ matrix \citep{sen2016contextual} or calculate second or third order tensors \citep{gopalan2016low}. 

%This is ok. I will write about col Exp3 and row WMA and why they are needed. Great. Also mention that previous works considered UCB1 and we do that as well.

%\todob{Discuss suitable choices for $\colalg$ and $\rowalg$.}


\section{Analysis}
\label{sec:analysis}
\begin{theorem}
\label{thm:LRB}
The cumulative regret upper bound for latent ranker algorithm is,
\begin{align*}
 R(n) = O\left(\frac{d \sqrt{L n}}{\Delta} + K \log n + K d \log d\right)
\end{align*}

where the gap $\Delta$ is the minimum gap between the optimal and the sub-optimal columns averaged over all users such that $\Delta =
  \min_{t \in [n]} \min_{J:\, J \neq J_\ast} \E\left[\tilde{r}_t(i, J_\ast)\right] - \E\left[\tilde{r}_t(i, J)\right]$.
\end{theorem}

%$n > L$ and $c$ is a constant such that $c > 2\sqrt{d}$

\begin{proof} \textbf{(Outline)}
The complete proof of Theorem \ref{thm:LRB} is given in Appendix \ref{sec:proof} and we sketch the main idea here.  We  express the total regret over all time steps as regret over time steps during which column algorithm suggests suboptimal arms and rest of the time steps. We can bound the contribution due to the former term as follows. The column learning algorithm has  low regret. This follows by an analysis similar to as in \cite{radlinski2008learning}. This implies that the column algorithm cannot suggest sub-optimal sets $J_t \neq J^*$ too often. 

The contribution to regret from the remaining time steps can be further decomposed as a sum over the users. We have a weighted majority algorithm for each user, and the regret bound follows by the standard guarantees as in \citet{littlestone1994weighted}.
\end{proof}

\begin{discussion}
\label{disc:proof1}
From the result in Theorem \ref{thm:LRB} we see that the regret consist of several parts. The first part of order $O\left(\frac{d \sqrt{L n}}{\Delta} \right)$ is the regret incurred for finding the $d$ best items (hott-topics) with high probability. The second part of order $O\left( K\log n + K d \log d\right)$ is incurred by WMA for finding the best permutation once the column MABs starts suggesting the $d$-best items. Note, that the result has the correct order as it \textit{does not} scale with $O\left(\sqrt{KLn}\right)$ like the independent user model algorithms.
\end{discussion}

\begin{discussion}
\label{disc:proof2}
Note, that for proving the regret bound we need an instance of LRA which uses EXP3 \citep{auer2002finite} as column MABs. This is because the feedbacks are no longer independent of each other. The feedback $f_{k,t}$ to the $k$-th column MAB$_k(n)$  is dependent on the feedback of $1$ to $k-1$-th column MABs. Hence, adversarial MABs which can work with any sequence of bounded feedback are required for giving theoretical guarantees in this setting.
\end{discussion}


\section{Experiments}
\label{sec:expt}
In this section, we compare LRA to several bandit algorithms in three experiments. The first two experiments are on synthetic dataset where all modeling assumptions hold. The third experiment is on a real-life dataset where we evaluate LRA when our modeling assumptions fail. In our first experiments user come uniform randomly while in experiment $2$ and $3$  users come in a Round Robin fashion over all time $[n]$. All results are averaged over $10$ independent random runs.

\textbf{Independent User Model Algorithms:} In this approach, each user has a separate version of base-bandit algorithm running independent of each other. As base-bandit algorithms we choose two variants of the ranked bandit algorithm (RBA) of \citet{radlinski2008learning}. The two variants of RBA uses two types of column learning algorithms, UCB1 \citep{auer2002finite} and EXP3 \citep{auer2002nonstochastic}, abbreviated as RBA-UCB1  and RBA-EXP3 respectively. EXP3 is a randomized algorithm suited for the adversarial setting while UCB1 is the standard algorithm used in the stochastic feedback setting. For RBA-UCB1, we choose the confidence interval at time $t$ as $c_{i, j}(t) = \sqrt{\frac{ \log t}{2N_{i, j}(t)}}$ for user $i$ and item $j$. Here, $N_{i, j}(t)$ denotes the number of times the $j$-th item has been observed by the $i$-th user base-bandit algorithm till timestep $t$. Note, that running independent vanilla UCB1 and EXP3 for every user is not feasible. This is because the vanilla versions are guaranteed to find a single best item for each user at rank $1$, while RBA-UCB1 and RBA-EXP3 will find a diverse list of $d$ best items for each user.

%it suggest previously unselected items by sampling uniform randomly at each time $t$. \todob{Why? This does not seem fair. We need to change baselines. I have an idea.}


\textbf{Matrix Completion Algorithms:} In the matrix completion approach, the algorithms try to reconstruct the user-item preference matrix $M$ from its noisy realization. We implement the widely used method to reconstruct partially observed noisy matrices, the non-negative matrix factorization. We term the corresponding algorithm as NMF Bandit (NMF-Ban). This algorithm is $\epsilon$-greedy in implementation whereby it reconstructs $M$ with $\epsilon$ probability and with $1-\epsilon$ probability it behaves greedily over the reconstructed matrix and suggest $d$ best item for the $i_t$-th user at every timestep $t$. Note, that NMF-Ban needs to sufficiently explore the noisy realizations of the matrix $M$ before it reconstructs. For this reason, we set the $\epsilon$ very low such that $\epsilon = 10^{-6}$.

%\todob{I thought that we do explore-exploit. In any case, say what $\epsilon$ / exploration horizon is.}

\textbf{Personalized Ranking Algorithms:} In this approach, we evaluate our proposed algorithm latent ranking bandit (LRA) by using three different types of column learning algorithms, EXP3, thompson sampling  \citep{thompson1933likelihood}, \citep{thompson1935theory}, \citep{agrawal2012analysis} and UCB1. We term them as LR-EXP3, LR-TS and LR-UCB1 respectively. Note, that thompson sampling is a Bayesian algorithm that performs better than UCB1 in stochastic setting due to its inherent prior assumptions on the distribution of the feedback. The row ranking components for all of these algorithms is the weighted majority algorithm (WMA) from \citet{littlestone1994weighted} which is suited for the full information setting. Note that we only show theoretical guarantees for LR-EXP3. We initialize the $k$-th column EXP3 with the column exploration parameter $\gamma_k = \sqrt{\frac{L \log L}{n}}$ as stated in \citet{auer2002nonstochastic}. Similarly, for LR-UCB1 we use  a confidence interval of $c_{k, j}(t) = \sqrt{\frac{\log t}{2N_{k,j}(t)}}$  for the $k$-th column MAB and $j$-th item. Here, $N_{k, j}(t)$ denotes the number of times the $j$-th item has been observed by the $k$-th column UCB1 algorithm till timestep $t$.

Note, that for all the versions of UCB1 used in several types of algorithms we use an aggressive confidence interval to enhance the performance of these  algorithms. This is contrary to the conservative confidence interval used to prove theoretical guarantees in \citet{auer2002finite}.

%\todob{$\gamma_k$ was never defined before. Why do you need to introduce it now?}

%\todob{The name of this setting is super confusing. If you say contextual, people expect feature vectors. We have no feature vectors. Say independent model for each user. That is clear and says precisely what we do.}
%
%\todosb{Changed this to Independent User Model Algorithms}

%\textbf{Matrix Completion Algorithms:} In the matrix completion approach, the algorithms try to reconstruct the user-item preference matrix $M$ from its noisy realization. We implement the widely used method to reconstruct partially observed noisy matrices, the non-negative matrix factorization. We term the corresponding algorithm as NMF Bandit (NMF-Ban). This algorithm is $\epsilon$-greedy in implementation whereby it reconstructs $M$ with $\epsilon$ probability and with $1-\epsilon$ probability it behaves greedily over the reconstructed matrix and suggest $d$ best item for the $i_t$-th user at every timestep $t$. 

%LinBan uses ridge regression to reconstruct $M$ from its estimated $d$-best columns while NMF-Ban uses matrix factorization to estimate the $U$ and $V$ matrix and reconstruct $M$ from its noisy realization. \todob{It is absolutely not clear what LinBan does.}

%\textbf{Personalized Ranking Algorithms:} In this approach, we evaluate our proposed algorithm Latent Ranking Bandit (LRA) by using two different types of Column MABs, EXP3 and UCB1. We term them as LREXP3 and LRUCB1 respectively. The row ranking components for both of these algorithms is still Weighted Majority Algorithm (WMA). Note that we only show theoretical guarantees for LREXP3. For LREXP3 we set the column exploration parameter $\gamma_k = \sqrt{\frac{L \log L}{n}}, \forall \text{ MAB}_k(n), k = 1,\dots, d$ and for LRUCB1 we use  a confidence interval of $c_{k, j}(t) = \sqrt{\frac{1.5 \log t}{N_{k,j}(t)}}$ for the $k$-th column MAB and $j$-th item.

%\todob{What? We do not do this anymore, no? Why is it here? It makes no sense to use a bandit algorithm since we are in the full-observation setting. No UCB1 please. Exp3 should be weighted majority.}

\textbf{Experiment 1:} This experiment is conducted to test the performance of LRA over small number of users and items. This simulated testbed consist of $500$ users, $50$ items, and rank$(M) = 2$. The vectors spanning $U$ and $V$, generating the user-item preference matrix $M$, are shown Figure \ref{fig:1}. The users are divided into a $70:30$ split such that $70\%$ of users prefer item $1$ and $30\%$ users prefer item  $2$. The item hott-topics are $V(1,:) = (0,1)$ and $V(2,:) = (1, 0)$ while remaining $70\%$ of items has feature $V(j',:) = (0.45, 0.55)$ and the rest have $V(j,:) = (0.55, 0.45)$. We create the user feature matrix $U$ similarly having a $70:30$ split such that $U(1,:) = (0,1)$, $U(2,:) = (0.2,0.8)$ and the remaining $70\%$ users having $U(i,:) = (0,0.8)$ and $30\%$ users having $U(i',:) = (0.7,0)$. The resulting matrix $M =UV^{\intercal}$ is such that algorithms that quickly find the easily identifiable hott-topics perform very well. From Figure \ref{fig:2} we can clearly see that LR-EXP3, LR-TS and LR-UCB1 outperforms all the other algorithms. Their regret curve flattens, indicating that they have learned the best items for each user.  NMF-Ban performs poorly as it fails to get a reasonable approximation of $M$ although it performs better than RBA-UCB1. Independent user model algorithms RBA-UCB1 and RBA-EXP3  perform poorly as the number of items per user is too large and the independent algorithms are not sharing information between them. 

%Although CTS performs well in this small testbed, its performance eventually degrades for larger environments. 

%\todob{Again, we never defined $j^\ast_1$ and $j^\ast_2$ before. Why do you think that an unexplained notation is better than explaining this in plain English?}


\begin{figure}[!th]
\centering
\begin{tabular}{cc}
\setlength{\tabcolsep}{0.1pt}
\subfigure[0.25\textwidth][Expt-$1$: $500$ Users, $50$ items, Rank $2$, User and Item vectors]
    %with $r_{i_{{i}\neq {*}}}=0.07$ and $r^{*}=0.1$
    {
    		\includegraphics[scale=0.11]{img/rank2_vec.png}
  		\label{fig:1}
    }
    &
    \subfigure[0.25\textwidth][Expt-$1$: Cumulative regret of different algorithms]
    %with $r_{i_{{i}\neq {*}}}=0.07$ and $r^{*}=0.1$
    {
    		\pgfplotsset{
		tick label style={font=\Large},
		label style={font=\Large},
		legend style={font=\Large},
		ylabel style={yshift=5pt},
		%legend style={legendshift=32pt},
		}
        \begin{tikzpicture}[scale=0.4]
      	\begin{axis}[
		xlabel={timestep},
		ylabel={Cumulative Regret},
		grid=major,
        %clip mode=individual,grid,grid style={gray!30},
        clip=true,
        %clip mode=individual,grid,grid style={gray!30},
        cycle list name=exotic,
  		legend style={at={(0.5,1.4)},anchor=north, legend columns=3} ]
      	% UCB
		%\addplot table{results/NewExpt1/Expt11/comp_subsampled_CTS0RR1S.txt};
		\addplot table{results/NewExpt2/Expt1/comp_subsampled_RBAEXP30RR1S.txt};
		\addplot table{results/NewExpt2/Expt1/comp_subsampled_LRTS0RR1S.txt};
		\addplot table{results/NewExpt2/Expt1/comp_subsampled_LRUCB0RR1S.txt};
		\addplot table{results/NewExpt2/Expt1/comp_subsampled_LREXP30RR1S.txt};
		\addplot table{results/NewExpt2/Expt1/comp_subsampled_NMF0RR1S.txt};
		%\addplot table{results/NewExpt1/Expt1/comp_subsampled_LinBan0RR1S.txt};
		%\addplot table{results/NewExpt1/Expt11/comp_subsampled_CUCB10RR1S.txt};
		\addplot table{results/NewExpt2/Expt1/comp_subsampled_RBAUCB10RR1S.txt};
		\legend{RBA-EXP3, LR-TS, LR-UCB1, LR-EXP3, NMF-Ban, RBA-UCB1} 
      	\end{axis}
      	\end{tikzpicture}
  		\label{fig:2}
    }
    \\
    \subfigure[0.25\textwidth][Expt-$2$: $1500$ Users, $100$ items, Rank $3$, User and Item vectors]
    %with $r_{i_{{i}\neq {*}}}=0.07$ and $r^{*}=0.1$
    {
    		\includegraphics[scale=0.11]{img/rank3_vec.png}
  		\label{fig:3}
    }
    &
    \subfigure[0.25\textwidth][Expt-$2$: Cumulative regret of different algorithms]
    %with $r_{i_{{i}\neq {*}}}=0.07$ and $r^{*}=0.1$
    {
    		\pgfplotsset{
		tick label style={font=\Large},
		label style={font=\Large},
		legend style={font=\Large},
		ylabel style={yshift=5pt},
		%legend style={legendshift=32pt},
		}
        \begin{tikzpicture}[scale=0.4]
      	\begin{axis}[
		xlabel={timestep},
		ylabel={Cumulative Regret},
		grid=major,
        %clip mode=individual,grid,grid style={gray!30},
        clip=true,
        %clip mode=individual,grid,grid style={gray!30},
        cycle list name=exotic,
  		legend style={at={(0.5,1.4)},anchor=north, legend columns=3} ]
      	% UCB
		%\addplot table{results/NewExpt1/Expt11/comp_subsampled_CTS0RR1S.txt};
		\addplot table{results/NewExpt2/Expt2/comp_subsampled_RBAEXP30RR1S.txt};
		\addplot table{results/NewExpt2/Expt2/comp_subsampled_LRTS0RR1S.txt};
		\addplot table{results/NewExpt2/Expt2/comp_subsampled_LRUCB0RR1S.txt};
		\addplot table{results/NewExpt2/Expt2/comp_subsampled_LREXP30RR1S.txt};
		\addplot table{results/NewExpt2/Expt2/comp_subsampled_NMF0RR1S.txt};
		%\addplot table{results/NewExpt1/Expt1/comp_subsampled_LinBan0RR1S.txt};
		%\addplot table{results/NewExpt1/Expt11/comp_subsampled_CUCB10RR1S.txt};
		\addplot table{results/NewExpt2/Expt2/comp_subsampled_RBAUCB10RR1S.txt};
		\legend{RBA-EXP3, LR-TS, LR-UCB1, LR-EXP3, NMF-Ban, RBA-UCB1} 
      	\end{axis}
      	\end{tikzpicture}
  		\label{fig:4}
    }
    \end{tabular}
    \caption{A comparison of the cumulative regret incurred by the various bandit algorithms. }
    \label{fig:karmed1}
    \vspace*{-1em}
\end{figure}

%\todob{Why two pictures? Why do you subindex some $V$ with $\ell_j$?}]
%\todob{Why two pictures? Why do you subindex some $V$ with $\ell_j$?}]

\textbf{Experiment 2:} We conduct the second experiment on a larger simulated database of $1500$ users, $100$ items and rank$(M)=3$. The vectors spanning $U$ and $V$, generating the user-item preference matrix $M$ is shown Figure \ref{fig:3}. The users are divided into a $60:30:10$ split such that $60\%$ of the users prefer item item $1$, $30\%$ prefer item $2$ and $10\%$ prefer item $3$.  Here, hott-topics are $V(1,:) = (1,0,0)$, $V(2,:) = (0,1, 0)$ and $V(3,:) = (0,0,1)$. The remaining $60\%$ of items have feature  $V(j,:) = (0.5, 0.25,0.25)$, $30\%$ have $V(j',:) = (0.25, 0.5, 0.25)$ and rest have $V(j^{''},:) = (0.25, 0.25, 0.5)$. We create the user feature matrix $U$ similarly having a $60:30:10$ split and the vectors spanning $U$ are only of the type that spans the simplex, i.e $U(i,:)=(1,0,0)$, $U(i',:)=(1,0,0)$ and $U(i^{''},:)=(1,0,0)$. Again, the resultant matrix $M =UV^{\intercal}$ is such that algorithms that quickly spot the easily identifiable hott-topics outperform others. From Figure \ref{fig:4} we can see that both LR-EXP3, LR-TS and LR-UCB1 again outperforms all the other algorithms. Their regret curve flattens much before all the other algorithms indicating that they have learned the best items for each user. The matrix completion algorithm NMF-Ban again fails to get a reasonable approximation of $M$ and performs poorly. Also, we see that both the independent user model algorithms RBA-UCB1 and RBA-EXP3 perform poorly as the number of users and the number of items per user is too large and the independent base-bandits (RBA) are not sharing information between themselves. In both the synthetic datasets, we see that stochastic column learning algorithm (UCB1) is outperforming adversarial column learning algorithm (EXP3) as the user preference over the best item is not changing over time. This has also been observed by \citet{radlinski2008learning}.

%which stems from the fact that all the user, 

%%%%%%% Rank 4 grouping %%%%
%[548, 99], [266, 93], [262, 96], [250, 28], [142, 59], [135, 79], [96, 88], [55, 78], [52, 9], [26, 52], [24, 76], [23, 67], [23, 68], [23, 95], [20, 97], [19, 66], [10, 40], [9, 7], [9, 85], [3, 1], [3, 17], [2, 61]


\textbf{Experiment 3:} We conduct the third experiment to test the performance of LRA when our modelling assumptions are violated. We use the Jester dataset \citep{goldberg2001eigentaste} which consist of over 4.1 million continuous ratings of 100 jokes from 73,421 users collected over 5 years. We sample randomly $2000$ users from this dataset and use singular value decomposition (SVD) to obtain a rank $4$ approximation of this user-joke rating matrix $M$. In the resultant matrix $M$, most of the users belong to the four classes preferring jokes 99, 93, 96 and 28, while a very small percentage of users prefer some other jokes. Note, that this condition results from the fact that this real-life dataset does not have the hott-topics structure. The rank $4$ approximation of $M$ of  is shown in Figure \ref{fig:5}, where we can clearly see the red stripes spanning the matrix indicating the low-rank structure of $M$. Furthermore, in this experiment we assume that the noise is independent Bernoulli over the entries of $M$ and hence this experiment deviates from our modeling assumptions. From \ref{fig:6} again we see that LR-EXP3, LR-TA and LR-UCB1 outperforms other algorithms. The regret curve of LRUCB1 does not flatten out which we attribute to the fact that LRUCB1 uses too large a confidence interval. The contextual and matrix completion algorithms perform significantly worse in this large testbed.

%\todob{The rank should be chosen using cross-validation. How do you know that rank $2$ is good?}

%[[1104, 99], [896, 93],[0,0][0,0]}

\begin{figure}[!th]
\centering
\begin{tabular}{cc}
\setlength{\tabcolsep}{0.1pt}
\subfigure[0.25\textwidth][Expt-$3$: $2000$ Users, $100$ items, Rank $4$ approximation of Jester Dataset]
    %with $r_{i_{{i}\neq {*}}}=0.07$ and $r^{*}=0.1$
    {
    \includegraphics[scale=0.08]{img/jester_rank4.png}
    	\label{fig:5}
    }
    &
\subfigure[0.25\textwidth][Expt-$3$: Cumulative regret of different algorithms]
    %with $r_{i_{{i}\neq {*}}}=0.07$ and $r^{*}=0.1$
    {
    		\pgfplotsset{
		tick label style={font=\Large},
		label style={font=\Large},
		legend style={font=\Large},
		ylabel style={yshift=5pt},
		%legend style={legendshift=32pt},
		}
        \begin{tikzpicture}[scale=0.4]
      	\begin{axis}[
		xlabel={timestep},
		ylabel={Cumulative Regret},
		grid=major,
        %clip mode=individual,grid,grid style={gray!30},
        clip=true,
        cycle list name=exotic,
        %clip mode=individual,grid,grid style={gray!30},
  		legend style={at={(0.5,1.4)},anchor=north, legend columns=3} ]
      	% UCB
		%\addplot table{results/NewExpt1/Expt11/comp_subsampled_CTS0RR1S.txt};
		\addplot table{results/NewExpt2/Expt3/comp_subsampled_RBAEXP30RR1S.txt};
		\addplot table{results/NewExpt2/Expt3/comp_subsampled_LRTS0RR1S.txt};
		\addplot table{results/NewExpt2/Expt3/comp_subsampled_LRUCB0RR1S.txt};
		\addplot table{results/NewExpt2/Expt3/comp_subsampled_LREXP30RR1S.txt};
		\addplot table{results/NewExpt2/Expt3/comp_subsampled_NMF0RR1S.txt};
		%\addplot table{results/NewExpt1/Expt1/comp_subsampled_LinBan0RR1S.txt};
		%\addplot table{results/NewExpt1/Expt11/comp_subsampled_CUCB10RR1S.txt};
		\addplot table{results/NewExpt2/Expt3/comp_subsampled_RBAUCB10RR1S.txt};
		\legend{RBA-EXP3, LR-TS, LR-UCB1, LR-EXP3, NMF-Ban, RBA-UCB1} 
      	\end{axis}
      	\end{tikzpicture}
  		\label{fig:6}
    }
 \end{tabular}
    \caption{A comparison of the cumulative regret in Jester Dataset }
    \label{fig:karmed}
    \vspace*{-1em}
\end{figure}

\section{Related Work}
\label{sec:related}
Previous works that have studied this setting have focused either on the rank-$1$ setting or have proposed solution where the underlying distributions are stochastic and having some structure.

\textbf{Rank-$1$ Setting:} The work of \citet{katariya2016stochastic} was proposed for a rank-$1$ bandit model with the assumption that the underlying distributions are stochastic. Similarly, \citet{katariya2017bernoulli} was proposed for the special case when the underlying distributions are Bernoulli.  A more simpler setting has also been studied in \citet{maillard2014latent}. All of these works used different variations of the Upper Confidence Bound (UCB) algorithm \cite{auer2002finite}, \citep{auer2010ucb} algorithm to construct a confidence interval set over row-column pairs to identify and eliminate sub-optimal rows and columns. These naturally results in algorithms that explore conservatively (for the sake of row and column elimination) and cannot work beyond the stochastic distribution assumption. 

%A more simpler setting has also been studied in \citet{maillard2014latent} under stochastic distribution assumption.


\textbf{Rank-$d$ Setting:} The work of \citet{kveton2017stochastic} can be viewed as a generalization of rank-$1$ bandits of \citet{katariya2016stochastic} to a higher rank of $d$. However, this work proposes a phase-based algorithm that calculates the square of the determinant of a $d\times d$ sub-matrix to eliminate sub-optimal rows and columns at the end of phases which is impractical for very large non-negative low-rank matrices. The theoretical guarantees hold for only stochastic distributions. Some other approaches involving non-negative matrix factorization \citet{sen2016contextual} or tensor based methods \citep{gopalan2016low} to reconstruct the matrix have also been proposed. These works require strong assumptions on the structure of the matrix such as all the matrices satisfy a weak statistical Restricted Isometric Property (RIP) or calculate third order tensors as in \citet{anandkumar2014tensor}. On the contrary, our simple and statistically efficient algorithm is easily generalizable to rank-$d$ and do not require any sort of costly matrix inversion or reconstruction operations or even row or column eliminations and hence are much easier to implement. 

%\textbf{•} Our approach is based on two key insights. \todoan{I wouldn't call these key insights. These are points for how we are different from earlier papers.}First, the earlier methods (like Upper Confidence Bound (UCB) algorithms, NMF-Bandits \citep{sen2016contextual}) are explicitly modeled on the stochastic i.i.d assumption on feedback and cannot perform well in non-stochastic settings. Moreover, their theoretical guarantees will also fail in non-stochastic setting. Hence, we need algorithms that can work on more generalized non-stochastic probability distribution settings. Secondly, we can formulate simple and computationally efficient algorithms that learn the best set of columns and best set of rows jointly with two separate non-stochastic bandit algorithm operating on rows and columns individually. These do not require any sort of costly matrix inversion or reconstruction operations or even row or column eliminations and hence are much easier to implement. 


%Previous works that have studied this setting have either proposed highly conservative algorithms or restricted themselves to a stricter set of assumptions. While \citet{katariya2016stochastic} was proposed for a rank $1$ bandit model with the assumption that the underlying distributions are stochastic, \citet{katariya2017bernoulli} was proposed for the special case when the underlying distributions are Bernoulli. Both these works used different variations of the phase-based UCB-Improved \citep{auer2010ucb} algorithm to construct a confidence interval set over row-column pairs to identify and eliminate sub-optimal rows and columns. These naturally results in algorithms that explore conservatively (for the sake of row and column elimination) and cannot work beyond the stochastic distribution assumption. Finally, \citet{kveton2017stochastic} can be viewed as a generalization of rank-$1$ bandits of \citet{katariya2016stochastic} to a higher rank of $d$. However, this work proposes a phase-based algorithm that calculates the square of the determinant of a $d\times d$ sub-matrix to eliminate sub-optimal rows and columns at the end of phases which is impractical for very large non-negative low-rank matrices. Some other approaches involving non-negative matrix factorization \citet{sen2016contextual} or tensor based methods \citep{gopalan2016low} to reconstruct the matrix have also been proposed. These works require strong assumptions on the structure of the matrix such as all the matrices satisfy a weak statistical Restricted Isometric Property (RIP) or calculate third order tensors as in \citet{anandkumar2014tensor}. A more simpler setting has also been studied in \citet{maillard2014latent}.

%\todob{The two paragraphs above and below should be moved to "Related Work''. They contain too many details that are not necessary to understand our design and contributions. Put "Related Work'' right before "Conclusions''. Also, the current comparison to prior work is a laundry list and completely inefficient. Better structure how we differ. Paragraph 1: Some people do only rank $1$. We do rank $d$. Paragraph 2: Most papers do stochastic. We do adversarial (with restrictions). Paragraph 3: Our main selling point is a simple algorithm that can be easily generalized beyond rank $1$. Focus on this difference.}

%are detailed in Section \ref{sec:related work}
 %\citet{sen2016contextual} is an online matrix completion algorithm which is an $\epsilon$-greedy algorithm that tries to reconstruct the matrix $M$ through non-negative matrix factorization. Note, that this approach requires that all the matrices satisfy a weak statistical Restricted Isometric Property, which is not always feasible in real life applications. Another approach is that of \citet{gopalan2016low} where the authors come up with an algorithm which uses the Robust Tensor Power (RTP) method of 
%\citet{anandkumar2014tensor} to reconstruct the matrix $M$, and then use the OFUL procedure of \citet{abbasi2011improved} to behave greedily over the reconstructed matrix. 
%But the RTP is a costly operation because the learner needs to construct a matrix of order $L\times L$ and $L\times L \times L$ to calculate the second and third order tensors for the reconstruction.  A more simpler setting has also been studied in \citet{maillard2014latent}

%Our approach is based on two key insights. \todoan{I wouldn't call these key insights. These are points for how we are different from earlier papers.}First, the earlier methods (like Upper Confidence Bound (UCB) algorithms, NMF-Bandits \citep{sen2016contextual}) are explicitly modeled on the stochastic i.i.d assumption on feedback and cannot perform well in non-stochastic settings. Moreover, their theoretical guarantees will also fail in non-stochastic setting. Hence, we need algorithms that can work on more generalized non-stochastic probability distribution settings. Secondly, we can formulate simple and computationally efficient algorithms that learn the best set of columns and best set of rows jointly with two separate non-stochastic bandit algorithm operating on rows and columns individually. These do not require any sort of costly matrix inversion or reconstruction operations or even row or column eliminations and hence are much easier to implement. 

\section{Conclusions and Future Directions}
\label{sec:conclusions}
In this paper, we studied the problem of finding the highest entry of a non-stochastic, non-negative low-rank matrix. We formulated the above problem as an online-learning problem and proposed the $\latentranker$ algorithm for this setting. We proved that an instance of algorithm has a regret bound in the special case of rank-$1$ setting that scales as $O\big(\frac{(\sqrt{L } + \sqrt{K }) \sqrt{n}}{\alpha}\big)$ and has the correct order with respect to rows, columns and rank of the row-column preference matrix $M$. We also evaluated our proposed algorithm on several simulated and real-life datasets and show that it outperforms the existing state-of-the-art algorithms. There are several directions where this work can be extended. Note that we only proved our theoretical results for the rank $1$ setting. Proving theoretical guarantees for $\latentranker$ algorithm will require additional assumptions on the structure of rewards and the matrix $M$. 

%Another interesting direction is to look at structures beyond hott-topics assumption on row and column matrix.

%There are several directions where this work can be extended. Note, that observing $d$ items at every timestep is helping LRA to learn more efficiently. Hence,  while keeping the hott-topics assumption it is worthwhile to study the personalized ranking setting when only $1$ item is allowed to be suggested at every timestep $t$. Another interesting direction is to look at structures where there are hott-topics assumption on user matrix as well as item matrix or maybe even at structures beyond hott-topics.


\newpage
\bibliographystyle{aaai}
\bibliography{biblio}


\appendix
%!TEX root = LatentBandits.tex

\clearpage
\onecolumn
\appendix

\section{Analysis}
\label{sec:analysis}

The reward for recommending $d$ columns $J$ to user $i$ is
\begin{align*}
  r_t(i, J) =
  \max \, \{\mu(k) \, r_t(i, J(k))\}_{k = 1}^d
\end{align*}
for weights $\mu(1) \geq \dots \geq \mu(d) > 0$. We also define an unweighted reward as
\begin{align*}
  \tilde{r}_t(i, J) =
  \max \, \{r_t(i, J(k))\}_{k = 1}^d\,.
\end{align*}
Let $J_\ast$ be the indices of hott topics and $J_{\ast, i}$ be their highest-reward permutation for user $i$. Let $J_t$ be our recommended columns at time $t$ and $J_{t, i}$ be their permutation for user $i$, which is computed by some row algorithm. The user at time $t$ is $i_t$. The expected $n$-step regret, where the randomness is only in the learning algorithm, is
\begin{align*}
  R(n) =
  \E\left[\sum_{t = 1}^n r_t(i_t, J_{\ast, i_t})\right] - \E\left[\sum_{t = 1}^n r_t(i_t, J_{t, i_t})\right]\,.
\end{align*}
The regret of the column learning algorithm in $n_0$ steps is bounded as
\begin{align*}
  \E\left[\sum_{t = 1}^{n_0} \tilde{r}_t(i, J_\ast)\right] - \E\left[\sum_{t = 1}^{n_0} \tilde{r}_t(i, J_t)\right] \leq
  d \sqrt{L n_0}
\end{align*}
for any $n_0$. Let
\begin{align*}
  \Delta = \min_{i \in [K], t \in [n]} \left(\tilde{r}_t(i, J_\ast) - \max_{J:\, J \neq J_\ast} \tilde{r}_t(i, J)\right)
\end{align*}
be the minimum gap. Then, based on the above inequalities, the probability that the column learning algorithm chooses $J_\ast$ at any time $t \geq n_0$ is bounded from below by
\begin{align}
  1 - \frac{d \sqrt{L n_0}}{\Delta n_0} =
  \frac{\Delta \sqrt{n_0} - d \sqrt{L}}{\Delta \sqrt{n_0}}
  \label{eq:opt lower bound}
\end{align}
for any $\Delta \geq d \sqrt{L / n_0}$.

Let $p_t$ be the probability that the column learning algorithm chooses $J_\ast$ at time $t$ and let $\tilde{J}_{\ast, t, i}$ be the permutation of $J_\ast$ for user $i$ at time $t$, according to our row algorithm. Then we can bound the regret from time $n_0$ as
\begin{align*}
  R(n)
  & = \sum_{i = 1}^K \sum_{t = n_0}^n \E\left[1\{i_t = i\} (r_t(i, J_{\ast, i}) - r_t(i, J_{t, i})\right] \\
  & = \sum_{i = 1}^K \sum_{t = n_0}^n \frac{1}{p_t} \E\left[1\{i_t = i, J_t = J_\ast\} (r_t(i, J_{\ast, i}) - r_t(i, J_{\ast, t, i})\right] \\
  & \leq \frac{\Delta \sqrt{n_0}}{\Delta \sqrt{n_0} - d \sqrt{L}} \sum_{i = 1}^K R_i(n)\,,
\end{align*}
where $R_i(n)$ is the expected $n$-step regret of the row algorithm in row $i$, conditioned on the fact that the column learning algorithm chooses $J_\ast$. One suitable row algorithm would be the weighted majority algorithm, which learns the optimal permutation for each $J$. Then $R_i(n) = O(\log n + \log d!) \approx O(\log n + d \log d)$.

In the first $n_0$ steps, we bound the regret trivially by $n_0$. It follows that the expected $n$-step regret is bounded by
\begin{align*}
  R(n) \leq
  n_0 + \frac{\Delta \sqrt{n_0}}{\Delta \sqrt{n_0} - d \sqrt{L}} K O(\log n + d \log d)\,.
\end{align*}


\end{document}
