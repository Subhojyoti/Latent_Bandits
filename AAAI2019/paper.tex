\def\year{2019}\relax
%File: formatting-instruction.tex
\documentclass[letterpaper]{article} %DO NOT CHANGE THIS
\usepackage{aaai19}  %Required
\usepackage{times}  %Required
\usepackage{helvet}  %Required
\usepackage{courier}  %Required
\usepackage{url}  %Required
\usepackage{graphicx}  %Required


%%%%%%%%%%%%% ADD CUSTOM PACKAGES TO macros.sty
\usepackage{macros}

%%%%%%%%%%%%%For adding comments

\newcommand{\todob}[2][]{\todo[color=cyan!20,size=\tiny,inline,#1]{B: #2}} % Brano's comments
\newcommand{\todosb}[2][]{\todo[color=green!20,size=\tiny,inline, #1]{S: #2}} %
\newcommand{\todoan}[2][]{\todo[color=black!20,size=\tiny,inline, #1]{A: #2}} %

\frenchspacing  %Required
\setlength{\pdfpagewidth}{8.5in}  %Required
\setlength{\pdfpageheight}{11in}  %Required
%PDF Info Is Required:
  \pdfinfo{
/Title (2019 Formatting Instructions for Authors Using LaTeX)
/Author (AAAI Press Staff)}
\setcounter{secnumdepth}{2}  
 \begin{document}
% The file aaai.sty is the style file for AAAI Press 
% proceedings, working notes, and technical reports.
%
\title{Latent Ranked Bandit}
\author{Author names withheld}
%\author{AAAI Press\\
%Association for the Advancement of Artificial Intelligence\\
%2275 East Bayshore Road, Suite 160\\
%Palo Alto, California 94303\\
%}
\maketitle
\begin{abstract}
We study the problem of learning personalized ranked lists of diverse items for multiple users, from sequential observations of user preferences. The user-item preference matrix is non-negative and low-rank. Existing methods for solving similar problems are based on reconstructing the preference matrix from its noisy observations using matrix factorization techniques, and typically require strong assumptions on the reconstructed matrix. We depart from this standard approach and focus on a family of low-rank matrices, where the set of most preferred items of all users is small and can be learned efficiently. Then we learn to present this set to each user in a personalized manner, in the order of the descending preferences of the user. We propose a computationally-efficient algorithm that implements this procedure, which we call latent ranker (LR), and prove a sublinear bound on its $n$-step regret. We evaluate the algorithm empirically on several synthetic and real-world datasets. In all experiments, we outperform existing state-of-the-art algorithms.
\end{abstract}

\section{Introduction}
\label{intro}
In this paper, we study the problem of recommending the best items to users who are coming sequentially. The learner has access to very less prior information about the users (cold start) and it has to adapt quickly to the user preferences and suggest the best item to each user. Furthermore, we consider a latent variable model such that the user-item preferences are generated as a mixture of user features and item features. Note, that only noisy observations from the user-item preference matrix is only visible to the learner and not the latent user or item features. The user-item preference matrix is low-rank in nature which is very common in recommender systems \citep{koren2009matrix}, \citep{ricci2011liorrokach}. Also, we assume that each user has a single best item preference.

	This complex problem can be conceptualized as a low rank online learning  problem where there are $K$ users and $L$ items. The reward matrix, denoted by $M\in [0,1]^{K\times L}$,  generating the rewards for user, item pair has a low rank structure. The online learning game proceeds as follows, at every timestep $t$,  nature reveals one user (or row) from $M$ where user is denoted by $i_t$. The learner selects $d$ items (or columns) from $[L]$, where an item is denoted by $j_t\in [L]$. Then the learner receives feedback $r_{t}(i_t,j_t)$ for all the $d$ items suggested from one noisy realization of $M(i_t,j_t)$ such that $\E[r_{t}(i_t,j_t)] = M(i_t,j_t)$. Then the goal of the learner is to minimize the cumulative regret , that is to minimize the total number of wrong items displayed over time to the user. Hence, the learner needs to quickly identify the best item $j^*$ for each $i\in [K]$ where $M(i,j^*) = \argmax _{j\in[L]} M(i,j)$. 
	
	This learning model can be conceptualized in the real-world scenario where the learner has  to suggest movies to users and each movie belongs to a different genre (say thriller, romance, comedy, etc). So, the learner can suggest $d$ movies belonging to different genres to each incoming user in a webpage, and the user can click one, or all, or none of the recommended movies (query abandonement).	
	
% where $Ber$ is a Binomial distribution over the entries in $M$
\subsection{Contributions}
\label{Contribution}
\input{Contributions}


\section{Setting}
\label{sec:setting}
%!TEX root = bandit_paper.tex

In this section, we study the online learning problem of finding the maximum entry of a non-stochastic, low-rank and non-negative matrix for the general rank-$d$ setting. 

\textbf{Hott-topics Assumption:} We focus on a family of low-rank matrices, which are known as hott topics. We define a \emph{hott-topics matrix} of rank $d$ as $M = U V\transpose$, where $U$ is a $K \times d$ non-negative matrix and $V$ is a $L \times d$ non-negative matrix that gives rise to the hott-topics structure. In particular, we assume that there exists $d$ rows $I^\ast$ in $U$ such that each row in $U$ can be represented as a convex combination of rows of $I^\ast$ and the zero vector. Hence, for an $A = \{a \in [0, 1]^{d \times 1}: \|a\|_1 \leq 1\}$ each row of $U$ can be expressed as,
\begin{align}
  \forall i \in [K] \ \exists \alpha \in A: U(I^\ast, :) \alpha = U(i, :)\,,
  \label{eq:hott topics1}
\end{align}
Similarly, we assume that there exist $d$ rows $J^\ast$ in $V$ such that each row of $V$ can be expressed as a convex combination of rows $J^\ast$ and the zero vector,
\begin{align}
  \forall j \in [L] \ \exists \alpha \in A: V(J^\ast, :) \alpha = V(j, :)\,,
  \label{eq:hott topics}
\end{align}
where $A = \{a \in [0, 1]^{d \times 1}: \|a\|_1 \leq 1\}$. Hence, the matrix $M$ represents preferences of users for items, $M(i, j)$ is the preference of user $i$ for item $j$. The rank $d$ of $M$ is the number of latent topics. The matrix $U$ are latent preferences of $K$ users over $d$ topics, where $U(i, :)$ are the preferences of user $i \in [K]$. The matrix $V$ are latent preferences of $L$ items in the space of $d$ topics, where $V(j, :)$ are the coordinates of item $j \in [L]$. Without loss of generality, we assume that $U \in [0, 1]^{K \times d}$ and $V \in [0, 1]^{L \times d}$. We assume that the coordinates are points in a simplex, that is $\|U(i, :)\|_1 \leq 1$ for all $i \in [K]$ and $\|V(j, :)\|_1 \leq 1$ for all $j \in [L]$. Note that our assumptions imply that $M(i, j) \geq 0$ for any $i \in [K]$ and $j \in [L]$.


\textbf{Rank-$d$ Setting:} Again, note that at time $t$, the preferences of users over items are encoded in a $K \times L$ \emph{preference matrix} $M_t = U_t V_t\transpose$, where $U_t$, and $V_t$ are defined as in \eqref{eq:hott topics1} and \eqref{eq:hott topics}. We assume that user and item preferences ($U_t$ and $V_t$ respectively) can change with time $t$. 
At every round $t$ the learner chooses $d$-pairs of rows and columns from $M_t$ denoted by $(I_t,J_t)\in \Pi_d([K])\times \Pi_d([L])$. It then observes all the values from the matrix $M_{t}(I_t,J_t)$ for all $i_t\in I_t$ and $j_t \in J_t$. The \emph{reward} for the agent for choosing arms $(I_t,J_t)$ at time $t$ is denoted by $r_t(i^\ast(I_t,J_t),j^\ast(I_t,J_t))$ such that,
\begin{align}
  (i^\ast(I,J),j^\ast(I,J)) = \argmax_{(i,j) \in (I\times J)} M_{t}(i,j)
  \label{eq:reward}
\end{align}

%a noisy \todob{We do not have any noise, right?} realization of

\todob{Say how the hott topics assumption simplifies the problem of finding the maximum entry of a matrix. Write it formally.}

%We study an online learning to rank problem, which we call a \emph{latent ranked bandit}. At time $t$, the preferences of users are encoded in a $K \times L$ \emph{preference matrix} $M_t = U_t V\transpose$, where $M$, $U_t$, and $V$ are defined as in \cref{sec:background}. We assume that user preferences $U_t$ can change with time $t$. A random user $i_t \in [K]$ arrives to the recommender system at time $t$ and we recommend $d$ items $J_t$ to this user. The \emph{reward} for recommending these items is $r_t(i_t, J_t)$, where
%\begin{align}
%  r_t(i, J) =
%  \max \, \{\mu(k) \, M_t(i, J(k)): k \in [d]\}
%  \label{eq:reward}
%\end{align}
%is the reward for recommending items $J$ to user $i$ at time $t$, $J(k)$ is the $k$-th item in $J$, and $\mu(k)$ is the weight of position $k \in [d]$. We assume that higher-ranked positions are more rewarding, $1 \geq \mu(1) \geq \dots \geq \mu(d) \geq 0$. The learning agent \emph{observes} the individual rewards of all recommended items, $M_t(i_t, J_t(k))$ for all $k \in [d]$.

%\todob{We need to motivate \eqref{eq:reward} from the user-modeling point of view. This should be the same motivation as in ranked bandits, except that $\mu$ enforces personalization, in the sense that the order matters.} \todoan{See the above comment. To motivate the fractional reward, how about saying that it can correspond to the length of the video the user watches. Like, we recommend a movie/video and if the user watches only half of the video, then the reward is 0.5.}

Since $U_t$ can change arbitrarily over time \todob{This obviously cannot be arbitrary. The assumption is that all $M_t$ have the same hott topics rows and columns. Write it formally.}, the reward in \eqref{eq:reward} is maximized by lists $J$ with highly rewarding items that are diverse, in the sense that they attain high rewards at different times $t \in [n]$. A remarkable property of our user-item preference matrices $M_t$ is that for any user $i \in [K]$ and any item $j \in [L]$ at any time $t$,
\begin{align*}
  \argmax_{(i, j) \in ([K] \times [L])} M_t(i, j) \in (I^\ast, J^\ast),
\end{align*}
where $I^\ast$ and $J^\ast$ is defined in \eqref{eq:hott topics1} and \eqref{eq:hott topics}. Therefore, it is possible to learn all potentially most rewarding pairs of rows and columns statistically efficiently.

%\todob{The definition of the regret below is incorrect for $d > 1$. We need to think about this. Actually, do we need the definition of the regret for $d > 1$ if we never bound it? We should use determinants and then intuitively explain what they mean. The rank $1$ case is intuitive and easy to explain.}

\textbf{Regret Definition (Rank-$d$):} Now we are ready to define our notion of optimality and regret for the general rank-$d$ scenario. Our goal is to minimize the expected $n$-step regret,
\begin{align}
  R(n) =
  \sum_{t = 1}^n \E\left[r_t(i^\ast_t, j^\ast_t) - r_t(i^\ast(I,J),j^\ast(I,J))\right]\,,
  \label{eq:regret1}
\end{align}
where the expectation is with respect to both randomly choosing rows $(I_t)$ and columns $(J_t)$ by the learning algorithm and potential randomness in the environment.

%Now we are ready to define our notion of optimality and regret. Let $J_\ast$ be the hott-topics items in \eqref{eq:hott topics} and $\pi_{\ast, i}$ be their permutation that maximizes the reward of user $i$ in hindsight,
%\begin{align*}
%  \pi_{\ast, i} =
%  \argmax_{\pi \in \Pi_d} \sum_{t = 1}^n r_t(i, \pi(J_\ast))\,.
%\end{align*}
%Let $J_t$ be our recommended items at time $t$ and $\pi_{t, i}$ be their permutation for user $i$, both of which are learned. Then our goal is to minimize the expected $n$-step regret,
%\begin{align}
%  R(n) =
%  \sum_{t = 1}^n \E\left[r_t(i_t, \pi_{\ast, i_t}(J_\ast)) - r_t(i_t, \pi_{t, i_t}(J_t))\right]\,,
%  \label{eq:regret}
%\end{align}
%where the expectation is with respect to both randomly arriving users and potential randomness in the learning algorithm.

%\todob{We do not really need the greedy definition of $J_\ast$ until the proof.}


%\todob{Why would you write contributions here? Contributions need to be clearly stated in Introduction.}

%\newpage
\section{Algorithm}
\label{sec:algorithm}
We propose the algorithm Latent Ranked Bandit, abbreviated as LRB (see Algorithm \ref{alg:latent-rank}) for solving the personalized ranking problem. This algorithm is motivated by the Ranked Bandit Algorithm (RBA) from \citet{radlinski2008learning} which is suited for finding a global ranking amongst all the users in the CBM click model. LRB is divided into two main components, the $d$ column MABs denoted by MAB$_1(n)$, MAB$_2(n), \dots,$ MAB$_d(n)$ and the $K$ row Weighted Majority Algorithms (WMA) for each user $[K]$. The WMA is motivated from \citet{littlestone1994weighted} which is suited for the total information setting. Note, that in our DCM setting all the clicks by the user is seen by the system and so it is a variation of total information setting. Each WMA consist of $d!$ arms for each of the permutations of rank $d$. Its main goal is to suggest a permutation $\Pi_{i_t}(S_t), \exists S_t \subseteq [L]$ such that the best item for the user $i_t$ is in rank $1$ of the permutation $\Pi_{i_t}(S_t)$.  LRB proceeds as follows, at every timestep $t\in[n]$ a user $i_t$ is revealed by nature, then the $d$ column MABs suggests columns $S_t = \lbrace {\ell}_{1}, {\ell}_{2},\dots, {\ell}_{d} \rbrace$ which it deems to be the $d$ best columns and by virtue of our setting, the $d$ hott-topics. If, there is any overlap in the suggestion, an arbitrary column is suggested which has not been selected before. Then, the row WMA for the $i_t$-th user selects a permutation $\Pi_{i_t}(S_t)$ by sampling through its distribution over the $d!$ arms and suggests a permutation $\tilde{S}_t = \lbrace \tilde{\ell}_{1}, \tilde{\ell}_{2},\dots, \tilde{\ell}_{d}\rbrace$ such that the item in rank $1$ is the best item for the $i_t$-th user with a high probability. Note, that we leave the implementation of the column MAB to the user which can be any stochastic or adversarial base-bandit algorithm discussed in Section \ref{related}.

Finally, after all the user clicks are recorded by the system both the column MABs and row WMA$_{i_t}(n)$ is updated. The $k$-th column bandit MAB$_k(n)$ is updated with feedback $f_{k,t} = \max_{j\in [k]} r_t(i_t, \ell_{j,t}) - \max_{j\in [k-1]} r_t(i_t,\ell_{j,t})$ such the monotonicity and submodularity properties discussed in section \ref{related} are maintained. Note that RBA is also a special case of submodular bandits such that $f_{k,t}\in\lbrace 0, 1\rbrace$ \citep{streeter2009online}. 


The row WMA$_{i_t}(n)$ update for the $i_t$-th user is quite straightforward as all the $d$ clicks are observed (total information). Hence, LRB can easily calculate the feedback for all the $d!$ permutation of $\Pi_{i_t}(S_t)$ and update its $d!$ arms representing each of those permutations. LRB calculates a weighted sum of feedback for each of the $d!$ permutation of $\Pi_{i_t}(S_t)$ and update the weights and its probabilities of the corresponding permutation. An illustrative diagram of the entire process is shown in Figure \ref{fig:rankedbandit}.

%Since, we observe all the clicks by user $i_t$ (total information),
\begin{figure}[!th]
    \includegraphics[scale=0.2]{img/RankedBand.png}
    \caption{Latent Ranked Bandit in rank $d=2$ scenario.}
    \label{fig:rankedbandit}
    \vspace*{-1em}
\end{figure}

\begin{algorithm}
\caption{Latent Ranked Bandit}
\label{alg:latent-rank}
  \begin{algorithmic}[1]
  \State \textbf{Input:} Rank $d$, horizon $n$.
  \State Initialize MAB$_1(n)$, MAB$_2(n), \dots,$ MAB$_d(n)$
  \State Initialize WMA$_1(n)$, WMA$_2(n), \dots,$ WMA$_K(n)$
    \For{$t = 1, \dots, n$}
      \State User $i_t$ comes to the system
      \For{$k = 1, \dots, d$}
      %\State // Choose $d$ items from $d$ column MABs
      \State ${\ell}_{k,t} \leftarrow$ suggest item $MAB_k(n)$
      \If{${\ell}_{k,t} \in \ell_{1, t},\dots,\ell_{k-1, t}$}
      \State ${\ell}_{k,t} \leftarrow$ Select arbitrary unselected item from $[L]\setminus \ell_{1, t},\dots,\ell_{k-1, t}$
      %\Else
      %\State $\ell_{k,t} \leftarrow \hat{\ell}_{k,t}$
      \EndIf
      \EndFor
      \State $\tilde{\ell}_{1,t},\tilde{\ell}_{2,t},\dots,\tilde{\ell}_{d,t}\leftarrow$ Permutation by WMA$_{i_t}(\ell_{1,t},\ell_{2,t},\dots,\ell_{d,t})$ by sampling  according to $p_{i_t,1},p_{i_t,2},\dots,p_{i_t,d!}$.
      %by sorting descendingly according to $w_{i_t,1},w_{i_t,2},\dots,w_{i_t,d}$.
      \State Present $\tilde{\ell}_{1,t},\tilde{\ell}_{2,t},\dots,\tilde{\ell}_{d,t}$ to user $i_t$ and record feedback $r_{t}(\tilde{\ell}_{1,t}), r_{t}(\tilde{\ell}_{2,t}),\dots,r_{t}(\tilde{\ell}_{d,t})$.
      \State Call Procedure UpdateColumnMAB
      \State Call Procedure UpdateRowWMA($i_t$)
    \EndFor
    \Procedure{UpdateColumnMAB}{}
    \For{$k = 1, \dots, d$}
    \State Update MAB$_k(n)$ with feedback $f_{k,t} = \max_{j\in [k]} r_t(i_t, \ell_{j,t}) - \max_{j\in [k-1]} r_t(i_t,\ell_{j,t})$
    % where $J_t[1: k] = \lbrace r_{t}({\ell}_{1,t}), r_{t}({\ell}_{2,t}),\dots,r_{t}({\ell}_{k,t})\rbrace$
    \EndFor
\EndProcedure
\Procedure{UpdateRowWMA}{$i_t$}
	\For{$k=1,2,\dots,d!$}
	\State $r_k = 0$ \Comment{Calculate weighted sum of rewards}
    \For{$j = 1, \dots, d$}
    \State $r_{k} = r_{k} + \frac{1}{j}r_{t}(\tilde{\ell}_{j,t})$
    \EndFor
    \State $w_{i_t,k} = w_{i_t,k} + r_k$  \Comment{Update weights}
    \EndFor
	\For{$k = 1, \dots, d!$}
	\State $p_{i_t,k} = \dfrac{\exp(w_{i_t,k})}{\sum_{b=1}^{d!} \exp(w_{i_t,b})} $ %\Comment{Update probabilities}
	\EndFor
\EndProcedure
  \end{algorithmic}
\end{algorithm}

\section{Analysis}
\label{analysis}
\begin{theorem}
\label{thm:LRB}
The cumulative regret upper bound for latent ranker algorithm is,
\begin{align*}
 R(n) = O\left(\frac{d \sqrt{L n}}{\Delta} + K \log n + K d \log d\right)
\end{align*}

where the gap $\Delta$ is the minimum gap between the optimal and the sub-optimal columns averaged over all users such that $\Delta =
  \min_{t \in [n]} \min_{J:\, J \neq J_\ast} \E\left[\tilde{r}_t(i, J_\ast)\right] - \E\left[\tilde{r}_t(i, J)\right]$.
\end{theorem}

%$n > L$ and $c$ is a constant such that $c > 2\sqrt{d}$

\begin{proof} \textbf{(Outline)}
The complete proof of Theorem \ref{thm:LRB} is given in Appendix \ref{sec:proof} and we sketch the main idea here.  We  express the total regret over all time steps as regret over time steps during which column algorithm suggests suboptimal arms and rest of the time steps. We can bound the contribution due to the former term as follows. The column learning algorithm has  low regret. This follows by an analysis similar to as in \cite{radlinski2008learning}. This implies that the column algorithm cannot suggest sub-optimal sets $J_t \neq J^*$ too often. 

The contribution to regret from the remaining time steps can be further decomposed as a sum over the users. We have a weighted majority algorithm for each user, and the regret bound follows by the standard guarantees as in \citet{littlestone1994weighted}.
\end{proof}

\begin{discussion}
\label{disc:proof1}
From the result in Theorem \ref{thm:LRB} we see that the regret consist of several parts. The first part of order $O\left(\frac{d \sqrt{L n}}{\Delta} \right)$ is the regret incurred for finding the $d$ best items (hott-topics) with high probability. The second part of order $O\left( K\log n + K d \log d\right)$ is incurred by WMA for finding the best permutation once the column MABs starts suggesting the $d$-best items. Note, that the result has the correct order as it \textit{does not} scale with $O\left(\sqrt{KLn}\right)$ like the independent user model algorithms.
\end{discussion}

\begin{discussion}
\label{disc:proof2}
Note, that for proving the regret bound we need an instance of LRA which uses EXP3 \citep{auer2002finite} as column MABs. This is because the feedbacks are no longer independent of each other. The feedback $f_{k,t}$ to the $k$-th column MAB$_k(n)$  is dependent on the feedback of $1$ to $k-1$-th column MABs. Hence, adversarial MABs which can work with any sequence of bounded feedback are required for giving theoretical guarantees in this setting.
\end{discussion}


%\newpage
\section{Experiments}
\label{expt}
In this section, we conduct three experiments and evaluate the performance of LRB against several bandit algorithms. Note, that at every timestep $t$, nature reveals an user $i_t$ and each algorithm suggests $d$ items to it and records all the $d$ feedbacks. The first two experiments are on simulated dataset where all our modelling assumptions hold. The third experiment is on a real-life dataset where we evaluate LRB when our modelling assumptions fail. In all our experiments users come in a Round Robin fashion over all time $[n]$. All the algorithms are averaged over $10$ independent runs.

\textbf{Contextual Algorithms:} In the contextual approach, each user has a separate version of base-bandit algorithm running independent of each other. As base-bandit algorithms we choose two versions of stochastic MAB, UCB1 and Thompson Sampling (TS), abbreviated as Contextual UCB1 (CUCB1) and Contextual TS (CTS) respectively. For UCB1, we choose the confidence interval at timestep $t$ as $c_{i_t, j}(t) = \sqrt{\frac{1.5 \log t}{N_{i_t,j}(t)}}$ for the $i_t$-th user and $j$-th item. Note, that both the vanilla UCB1 and TS is used to find the best item for each user at rank $1$, while for the remaining positions $k= 2,\dots, d$ it  suggest previously unselected items by sampling uniform randomly at every timestep $t$. 

\textbf{Matrix Completion Algorithms:} In the matrix completion approach, the algorithms try to reconstruct the user-item preference matrix $M$ from its noisy realization. We use two widely used method to reconstruct partially observed noisy matrices, linear ridge regression and non-negative matrix factorization. We term the corresponding algorithms as Linear Bandit (LinBan) and NMF Bandit (NMF-Ban) respectively. Both of these algorithms are $\epsilon$-greedy in implementation whereby they reconstruct $M$ with $\epsilon$ probability and with $1-\epsilon$ probability they behave greedily over the reconstructed matrix and suggest $d$ best item for the $i_t$-th user at every timestep $t$. LinBan uses ridge regression to reconstruct $M$ from its estimated $d$-best columns while NMF-Ban uses matrix factorization to estimate the $U$ and $V$ matrix and reconstruct $M$.

\textbf{Personalized Ranking Algorithms:} In this approach, we evaluate our proposed algorithm Latent Ranking Bandit (LRB) by using two different types of base-bandits, EXP3 and UCB1 as column bandits. We term them as LREXP3 and LRUCB1 respectively. For LREXP3 we set $\gamma = \sqrt{\frac{L \log L}{n}}$ and for UCB1 we use  a confidence interval of $c_{k, j}(t) = \sqrt{\frac{1.5 \log t}{N_{k,j}(t)}}$ for the $k$-th column bandit and $j$-th item.

\textbf{Experiment 1:} This experiment is conducted to test the performance of LRB over  small number of users and items. This simulated testbed consist of $500$ users, $50$ items and rank$(M) = 2$. The vectors spanning $U$ and $V$, generating the user-item preference matrix $M$ is shown Figure \ref{fig:1}. The users are divided into a $70:30$ split such that $70\%$ over users prefer item $j^*_1$ over $j^*_2$.  From Figure \ref{fig:2} we can clearly see that both LREXP3 and LRUCB1 outperforms all the other algorithms. Their regret curve flattens, indicating that they have learned the best items for each user.  NMF-Ban and LinBan has nearly similar performance as both of these algorithms fail to get a reasonable approximation of $M$ although they perform better than CUCB1. Also,  CUCB1 performs poorly as the number of items per user is too large and the gaps are also small. Although CTS performs well in this small testbed, its performance eventually degrades for larger environments. 


\begin{figure}[!th]
\centering
\begin{tabular}{cc}
\setlength{\tabcolsep}{0.1pt}
\subfigure[0.25\textwidth][Expt-$1$: $500$ Users, $50$ items, Rank $2$, User and Item vectors]
    %with $r_{i_{{i}\neq {*}}}=0.07$ and $r^{*}=0.1$
    {
    		\includegraphics[scale=0.11]{img/rank2_vec.png}
  		\label{fig:1}
    }
    &
    \subfigure[0.25\textwidth][Expt-$1$: Cumulative regret of different algorithms]
    %with $r_{i_{{i}\neq {*}}}=0.07$ and $r^{*}=0.1$
    {
    		\pgfplotsset{
		tick label style={font=\Large},
		label style={font=\Large},
		legend style={font=\Large},
		ylabel style={yshift=5pt},
		%legend style={legendshift=32pt},
		}
        \begin{tikzpicture}[scale=0.4]
      	\begin{axis}[
		xlabel={timestep},
		ylabel={Cumulative Regret},
		grid=major,
        %clip mode=individual,grid,grid style={gray!30},
        clip=true,
        %clip mode=individual,grid,grid style={gray!30},
        cycle list name=exotic,
  		legend style={at={(0.5,1.4)},anchor=north, legend columns=3} ]
      	% UCB
		\addplot table{results/NewExpt1/Expt1/comp_subsampled_CTS0RR1S.txt};
		\addplot table{results/NewExpt1/Expt1/comp_subsampled_LRUCB0RR1S.txt};
		\addplot table{results/NewExpt1/Expt1/comp_subsampled_LREXP30RR1S.txt};
		\addplot table{results/NewExpt1/Expt1/comp_subsampled_NMF0RR1S.txt};
		\addplot table{results/NewExpt1/Expt1/comp_subsampled_LinBan0RR1S.txt};
		\addplot table{results/NewExpt1/Expt1/comp_subsampled_CUCB10RR1S.txt};
		\legend{CTS, LRUCB1, LREXP3, NMF-Ban, LinBan, CUCB1} 
      	\end{axis}
      	\end{tikzpicture}
  		\label{fig:2}
    }
    \\
    \subfigure[0.25\textwidth][Expt-$2$: $1500$ Users, $100$ items, Rank $3$, User and Item vectors]
    %with $r_{i_{{i}\neq {*}}}=0.07$ and $r^{*}=0.1$
    {
    		\includegraphics[scale=0.11]{img/rank3_vec.png}
  		\label{fig:3}
    }
    &
    \subfigure[0.25\textwidth][Expt-$2$: Cumulative regret of different algorithms]
    %with $r_{i_{{i}\neq {*}}}=0.07$ and $r^{*}=0.1$
    {
    		\pgfplotsset{
		tick label style={font=\Large},
		label style={font=\Large},
		legend style={font=\Large},
		ylabel style={yshift=5pt},
		%legend style={legendshift=32pt},
		}
        \begin{tikzpicture}[scale=0.4]
      	\begin{axis}[
		xlabel={timestep},
		ylabel={Cumulative Regret},
		grid=major,
        %clip mode=individual,grid,grid style={gray!30},
        clip=true,
        %clip mode=individual,grid,grid style={gray!30},
        cycle list name=exotic,
  		legend style={at={(0.5,1.4)},anchor=north, legend columns=3} ]
      	% UCB
		\addplot table{results/NewExpt1/Expt2/comp_subsampled_CTS0RR1S.txt};
		\addplot table{results/NewExpt1/Expt2/comp_subsampled_LRUCB0RR1S.txt};
		\addplot table{results/NewExpt1/Expt2/comp_subsampled_LREXP30RR1S.txt};
		\addplot table{results/NewExpt1/Expt2/comp_subsampled_NMF0RR1S.txt};
		\addplot table{results/NewExpt1/Expt2/comp_subsampled_LinBan0RR1S.txt};
		\addplot table{results/NewExpt1/Expt2/comp_subsampled_CUCB10RR1S.txt};
		\legend{CTS, LRUCB1, LREXP3, NMF-Ban, LinBan, CUCB1} 
      	\end{axis}
      	\end{tikzpicture}
  		\label{fig:4}
    }
    \end{tabular}
    \caption{A comparison of the cumulative regret incurred by the various bandit algorithms. }
    \label{fig:karmed1}
    \vspace*{-1em}
\end{figure}

\textbf{Experiment 2:} We conduct the second experiment on a larger simulated database of $1500$ users, $100$ items and rank$(M)=3$. The vectors spanning $U$ and $V$, generating the user-item preference matrix $M$ is shown Figure \ref{fig:3}. The users are divided into a $60:30:10$ split such that $60\%$ of the users prefer item $j^*_1$, $30\%$ prefer $j^*_2$ and $10\%$ prefer $j_3^*$.  The vectors spanning $U$ are only of the type that spans the simplex. From Figure \ref{fig:4} we can see that both LREXP3 and LRUCB1 again outperforms all the other algorithms. Their regret curve flattens much before all the other algorithms indicating that they have learned the best items for each user. Again, the matrix completion algorithms NMF-Ban and LinBan fail to get a reasonable approximation of $M$ and perform poorly. Also, we see that both the contextual algorithms CUCB1 and CTS perform poorly as the number of users and the number of items per user is too large and the independent base-bandits are not sharing information between themselves.  

%which stems from the fact that all the user, 

\textbf{Experiment 3:} We conduct the third experiment to test the performance of LRB when our modelling assumptions are violated. We use the Jester dataset \citep{goldberg2001eigentaste} which consist of over 4.1 million continuous ratings of 100 jokes from 73,421 users collected over 5 years. We sample randomly around 2000 users from this dataset and use singular value decomposition (SVD) to obtain a rank $2$ approximation of this user-joke rating matrix $M$. The rank $2$ approximation of $M$ of  is shown in Figure \ref{fig:5}, where we can clearly see the red stripes spanning the matrix indicating the low-rank structure of $M$. Furthermore, in this experiment we assume that the noise is independent Bernoulli over the entries of $M$ and hence this experiment deviates from our modeling assumptions. From \ref{fig:6} again we see that LREXP3 outperforms other algorithms. The regret curve of LRUCB1 does not flatten out which we attribute to the fact that LRUCB1 uses too large a confidence interval. The contextual and matrix completion algorithms perform significantly worse in this large testbed.



%[[1104, 99], [896, 93],[0,0][0,0]}

\begin{figure}[!th]
\centering
\begin{tabular}{cc}
\setlength{\tabcolsep}{0.1pt}
\subfigure[0.25\textwidth][Expt-$3$: $2000$ Users, $100$ arms, Rank $2$ approximation of Jester Dataset]
    %with $r_{i_{{i}\neq {*}}}=0.07$ and $r^{*}=0.1$
    {
    \includegraphics[scale=0.08]{img/jester_rank2.png}
    	\label{fig:5}
    }
    &
\subfigure[0.25\textwidth][Expt-$3$: Cumulative regret of different algorithms]
    %with $r_{i_{{i}\neq {*}}}=0.07$ and $r^{*}=0.1$
    {
    		\pgfplotsset{
		tick label style={font=\Large},
		label style={font=\Large},
		legend style={font=\Large},
		ylabel style={yshift=5pt},
		%legend style={legendshift=32pt},
		}
        \begin{tikzpicture}[scale=0.4]
      	\begin{axis}[
		xlabel={timestep},
		ylabel={Cumulative Regret},
		grid=major,
        %clip mode=individual,grid,grid style={gray!30},
        clip=true,
        cycle list name=exotic,
        %clip mode=individual,grid,grid style={gray!30},
  		legend style={at={(0.5,1.4)},anchor=north, legend columns=3} ]
      	% UCB
		
		\addplot table{results/NewExpt1/Expt3/comp_subsampled_CTS0RR1S.txt};
		\addplot table{results/NewExpt1/Expt3/comp_subsampled_LRUCB0RR1S.txt};
		\addplot table{results/NewExpt1/Expt3/comp_subsampled_LREXP30RR1S.txt};
		\addplot table{results/NewExpt1/Expt3/comp_subsampled_NMF0RR1S.txt};
		\addplot table{results/NewExpt1/Expt3/comp_subsampled_LinBan0RR1S.txt};
		\addplot table{results/NewExpt1/Expt3/comp_subsampled_CUCB10RR1S.txt};
		\legend{CTS, LRUCB1, LREXP3, NMF-Ban, LinBan, CUCB1} 
      	\end{axis}
      	\end{tikzpicture}
  		\label{fig:6}
    }
 \end{tabular}
    \caption{A comparison of the cumulative regret in Jester Dataset }
    \label{fig:karmed}
    \vspace*{-1em}
\end{figure}

\section{Related Work}
\label{related}
Our work lies at the intersection of several existing areas of research, which we survey below. 
\todoan{This can be compressed a lot. For one, if a there are just 1-2 papers using something, that shouldn't be part of a classification. This is the case, for example, when you say literature for latent bandits can be classified. You are describing things in too much detail. You can just say the setting which the other papers deal and why they they are different than this paper etc}

\textbf{Latent Bandits:} The existing algorithms in latent bandit literature can be broadly classified into two groups: the online matrix completion algorithms and the independent user model algorithms. The \textit{online matrix completion algorithms} try to reconstruct the user-item preference matrix $M$ from a noisy realization combining different approaches of online learning algorithms and matrix factorization algorithms. 
The NMF-Bandit algorithm in \citet{sen2016contextual} is an online matrix completion algorithm which is an $\epsilon$-greedy algorithm that tries to reconstruct the matrix $M$ through non-negative matrix factorization. Note, that this approach requires that all the matrices satisfy a weak statistical Restricted Isometric Property, which is not always feasible in real life applications. Another approach is that of \citet{gopalan2016low} where the authors come up with an algorithm which uses the Robust Tensor Power (RTP) method of 
\citet{anandkumar2014tensor} to reconstruct the matrix $M$, and then use the OFUL procedure of \citet{abbasi2011improved} to behave greedily over the reconstructed matrix. 
But the RTP is a costly operation because the learner needs to construct a matrix of order $L\times L$ and $L\times L \times L$ to calculate the second and third order tensors for the reconstruction.  A more simpler setting has also been studied in \citet{maillard2014latent} where all the users tend to come from only one class and hence this approach is also not quite realistic. 

The second type of algorithms are the \textit{independent user model algorithms} where for each user $i\in[K]$ a separate instance of a base-bandit algorithm is implemented to find the best item for the user. These base -bandits run independent of each other without sharing any information. These can be randomized algorithms suited for the adversarial setting like EXP3 \citep{auer2002nonstochastic} or UCB type algorithms suited for the stochastic setting llike UCB1 \citep{auer2002finite}, MOSS \citep{audibert2009minimax}, OCUCB \citep{lattimore2015optimally}, KL-UCB \citep{cappe2013kullback}, \citep{garivier2011kl} or even Bayesian algorithms like Thompson Sampling \citep{thompson1933likelihood}, \citep{thompson1935theory}, \citep{agrawal2012analysis}.

%can be used for this purpose.
%which are a set of frequentist indexed based algorithms
% Several powerful variation of the stochastic or non-stochastic multi-armed bandit algorithms can be used for this purpose.
%method with $\epsilon$ probability or with $1-\epsilon$ it behaves greedily over the already reconstructed matrix $\hat{M}$.

\textbf{Ranked Bandits:} Bandits have been used to rank items for online recommendations where the goal is is to present a list of $d$ items out of $L$ that maximizes the satisfaction of the user. A popular approach is to model each of the $d$ rank positions as a Multi Armed Bandit (MAB) problem and use a base-bandit algorithm to solve it. This was first proposed in \citet{radlinski2008learning} which  showed that query abandonment by user can also be successfully used to learn rankings. Later works on ranking such as \citet{slivkins2010ranked} and \citet{slivkins2013ranked} uses additional assumptions to handle  exponentially large number of items such that items and user models lie within a metric space and satisfy Lipschitz condition. 

\textbf{Ranking in Click Models:} Several algorithms have been proposed to solve the ranking problem in specific click models. Popular click models that have been studied extensively are Document Click Model (DCM), Position Based Click Model (PBM) and Cascade Click Model (CBM). For a survey of existing click models a reader may look into \citet{chuklin2015click}. While \citet{katariya2017bernoulli}, \citet{katariya2016stochastic} works in PBM, \citet{zoghi2017online} works in both PBM and CBM. Finally, \citet{kveton2017stochastic} can be viewed as a generalization of rank-1 bandits of \citet{katariya2016stochastic} to a higher rank. Note, that the theoretical guarantees of these algorithms does not hold beyond the specific click models.


\textbf{Online Sub-modular maximization:} Maximization of submodular functions has wide applications in machine learning, artificial
intelligence and in recommender systems \citep{nemhauser1978analysis}, \citep{krause2014submodular}. A submodular function $f : 2^V \rightarrow \mathbb{R}$ for a finite ground set $V$ is a set function that assign each
subset $S \subseteq V$ a value $f(S)$. We define the gain of the function $f$ as $G_f(e|S) = f(S \cup \lbrace e\rbrace) - f(S)$ where the element $\lbrace e\rbrace \in V\setminus S$ and $S \subseteq V$. Also, $f$ satisfies the following two criteria:-
\begin{enumerate}
\item Monotonicity: A set function $f : 2^V \rightarrow \mathbb{R}$ is monotone if for every $A \subseteq B \subseteq V, f(A) \leq f(B)$.
\item Submodularity: A set function $f : 2^V \rightarrow \mathbb{R}$ is submodular if for every $A \subseteq B \subseteq V$ and $\lbrace e\rbrace \in V \setminus B$ it holds that $G_f(e | A) \geq G_f(e | B)$.
\end{enumerate}

Intuitively, a submodular function states that after performing a set $A$ of actions, the marginal gain of another action $e$ does not increase the gain for performing other actions in $B \setminus A$. Online submodular function maximization has been studied in \citet{streeter2009online} where the authors propose a general algorithm whereas  \citet{radlinski2008learning} can be considered as special case of it when the payoff is only between $\lbrace 0, 1\rbrace$. Also, in the contextual feature based setup online  submodular maximization has been studied by  \citet{yue2011linear}. An interesting property of submodular function is that a greedy algorithm using it is guaranteed to perform atleast $\left( 1 - \frac{1}{e}\right)$ of the optimal algorithm and this factor $\left( 1 - \frac{1}{e}\right)$ is not improvable by any polynomial time algorithm \citep{nemhauser1978analysis}.




\section{Conclusions and Future Directions}
\label{conclusions}
In this paper, we studied the problem of finding the highest entry of a non-stochastic, non-negative low-rank matrix. We formulated the above problem as an online-learning problem and proposed the $\latentranker$ algorithm for this setting. We proved that an instance of algorithm has a regret bound in the special case of rank-$1$ setting that scales as $O\big(\frac{(\sqrt{L } + \sqrt{K }) \sqrt{n}}{\alpha}\big)$ and has the correct order with respect to rows, columns and rank of the row-column preference matrix $M$. We also evaluated our proposed algorithm on several simulated and real-life datasets and show that it outperforms the existing state-of-the-art algorithms. There are several directions where this work can be extended. Note that we only proved our theoretical results for the rank $1$ setting. Proving theoretical guarantees for $\latentranker$ algorithm will require additional assumptions on the structure of rewards and the matrix $M$. 

%Another interesting direction is to look at structures beyond hott-topics assumption on row and column matrix.

%There are several directions where this work can be extended. Note, that observing $d$ items at every timestep is helping LRA to learn more efficiently. Hence,  while keeping the hott-topics assumption it is worthwhile to study the personalized ranking setting when only $1$ item is allowed to be suggested at every timestep $t$. Another interesting direction is to look at structures where there are hott-topics assumption on user matrix as well as item matrix or maybe even at structures beyond hott-topics.




\newpage
\bibliographystyle{aaai}
\bibliography{biblio}


\appendix
%!TEX root = LatentBandits.tex

\clearpage
\onecolumn
\appendix

\section{Analysis}
\label{sec:analysis}

The reward for recommending $d$ columns $J$ to user $i$ is
\begin{align*}
  r_t(i, J) =
  \max \, \{\mu(k) \, r_t(i, J(k))\}_{k = 1}^d
\end{align*}
for weights $\mu(1) \geq \dots \geq \mu(d) > 0$. We also define an unweighted reward as
\begin{align*}
  \tilde{r}_t(i, J) =
  \max \, \{r_t(i, J(k))\}_{k = 1}^d\,.
\end{align*}
Let $J_\ast$ be the indices of hott topics and $J_{\ast, i}$ be their highest-reward permutation for user $i$. Let $J_t$ be our recommended columns at time $t$ and $J_{t, i}$ be their permutation for user $i$, which is computed by some row algorithm. The user at time $t$ is $i_t$. The expected $n$-step regret, where the randomness is only in the learning algorithm, is
\begin{align*}
  R(n) =
  \E\left[\sum_{t = 1}^n r_t(i_t, J_{\ast, i_t})\right] - \E\left[\sum_{t = 1}^n r_t(i_t, J_{t, i_t})\right]\,.
\end{align*}
The regret of the column learning algorithm in $n_0$ steps is bounded as
\begin{align*}
  \E\left[\sum_{t = 1}^{n_0} \tilde{r}_t(i, J_\ast)\right] - \E\left[\sum_{t = 1}^{n_0} \tilde{r}_t(i, J_t)\right] \leq
  d \sqrt{L n_0}
\end{align*}
for any $n_0$. Let
\begin{align*}
  \Delta = \min_{i \in [K], t \in [n]} \left(\tilde{r}_t(i, J_\ast) - \max_{J:\, J \neq J_\ast} \tilde{r}_t(i, J)\right)
\end{align*}
be the minimum gap. Then, based on the above inequalities, the probability that the column learning algorithm chooses $J_\ast$ at any time $t \geq n_0$ is bounded from below by
\begin{align}
  1 - \frac{d \sqrt{L n_0}}{\Delta n_0} =
  \frac{\Delta \sqrt{n_0} - d \sqrt{L}}{\Delta \sqrt{n_0}}
  \label{eq:opt lower bound}
\end{align}
for any $\Delta \geq d \sqrt{L / n_0}$.

Let $p_t$ be the probability that the column learning algorithm chooses $J_\ast$ at time $t$ and let $\tilde{J}_{\ast, t, i}$ be the permutation of $J_\ast$ for user $i$ at time $t$, according to our row algorithm. Then we can bound the regret from time $n_0$ as
\begin{align*}
  R(n)
  & = \sum_{i = 1}^K \sum_{t = n_0}^n \E\left[1\{i_t = i\} (r_t(i, J_{\ast, i}) - r_t(i, J_{t, i})\right] \\
  & = \sum_{i = 1}^K \sum_{t = n_0}^n \frac{1}{p_t} \E\left[1\{i_t = i, J_t = J_\ast\} (r_t(i, J_{\ast, i}) - r_t(i, J_{\ast, t, i})\right] \\
  & \leq \frac{\Delta \sqrt{n_0}}{\Delta \sqrt{n_0} - d \sqrt{L}} \sum_{i = 1}^K R_i(n)\,,
\end{align*}
where $R_i(n)$ is the expected $n$-step regret of the row algorithm in row $i$, conditioned on the fact that the column learning algorithm chooses $J_\ast$. One suitable row algorithm would be the weighted majority algorithm, which learns the optimal permutation for each $J$. Then $R_i(n) = O(\log n + \log d!) \approx O(\log n + d \log d)$.

In the first $n_0$ steps, we bound the regret trivially by $n_0$. It follows that the expected $n$-step regret is bounded by
\begin{align*}
  R(n) \leq
  n_0 + \frac{\Delta \sqrt{n_0}}{\Delta \sqrt{n_0} - d \sqrt{L}} K O(\log n + d \log d)\,.
\end{align*}


\end{document}
