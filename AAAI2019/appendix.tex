%!TEX root = LatentBandits.tex

\clearpage
\onecolumn
\appendix

\section{Analysis}
\label{sec:analysis}

The reward for recommending $d$ columns $J$ to user $i$ is
\begin{align*}
  r_t(i, J) =
  \max \, \{\mu(k) \, r_t(i, J(k))\}_{k = 1}^d
\end{align*}
for weights $\mu(1) \geq \dots \geq \mu(d) > 0$. We also define an unweighted reward as
\begin{align*}
  \tilde{r}_t(i, J) =
  \max \, \{r_t(i, J(k))\}_{k = 1}^d\,.
\end{align*}
Let $J_\ast = (1, \dots, d)$ be the indices of the hott topics. Let $J_{\ast, i}$ be the permutation of $J_\ast$ that is optimal for user $i$. Let $J_t$ be our recommended columns at time $t$ and $J_{t, i}$ be their permutation for user $i$, which is computed by some later specified strategy. The user at time $t$ is $i_t$. The expected $n$-step regret, where the randomness is in the learning algorithm, is
\begin{align*}
  R(n) =
  \E\left[\sum_{t = 1}^n r_t(i_t, J_{\ast, i_t})\right] - \E\left[\sum_{t = 1}^n r_t(i_t, J_{t, i_t})\right]\,.
\end{align*}
The regret of the column learning algorithm in $n_0$ steps is bounded as
\begin{align*}
  \E\left[\sum_{t = 1}^{n_0} \tilde{r}_t(i, J_\ast)\right] - \E\left[\sum_{t = 1}^{n_0} \tilde{r}_t(i, J_t)\right] \leq
  d \sqrt{L n_0}
\end{align*}
for any $n_0$. Let
\begin{align*}
  \Delta = \min_{i \in [K], t \in [n]} \left(\tilde{r}_t(i, J_\ast) - \max_{J:\, J \neq J_\ast} \tilde{r}_t(i, J)\right)
\end{align*}
be the minimum gap. Then based on the above inequalities, the probability that the column learning algorithm chooses $J_\ast$ at any time $t \geq n_0$, is bounded from below by
\begin{align*}
  1 - \frac{d \sqrt{L n_0}}{\Delta n_0} =
  \frac{\Delta \sqrt{n_0} - d \sqrt{L}}{\Delta \sqrt{n_0}}\,.
  \label{eq:opt lower bound}
\end{align*}
Let $p_t$ be the probability that the column learning algorithm chooses $J_\ast$ at time $t$ and let $\tilde{J}_{\ast, t, i}$ be the permutation of $J_\ast$ for user $i$ at time $t$, according to our row algorithm. Then we can bound the regret from time $n_0$ as
\begin{align*}
  R(n)
  & = \sum_{i = 1}^K \sum_{t = n_0}^n \E\left[1\{i_t = i\} (r_t(i, J_{\ast, i}) - r_t(i, J_{t, i})\right] \\
  & = \sum_{i = 1}^K \sum_{t = n_0}^n \frac{1}{p_t} \E\left[1\{i_t = i, J_t = J_\ast\} (r_t(i, J_{\ast, i}) - r_t(i, J_{\ast, t, i})\right] \\
  & \leq \frac{\Delta \sqrt{n_0}}{\Delta \sqrt{n_0} - d \sqrt{L}} \sum_{i = 1}^K R_i(n)\,,
\end{align*}
where $R_i(n)$ is the expected $n$-step regret of the row algorithm in row $i$, conditioned on the fact that the column learning algorithm chooses $J_\ast$. One suitable row algorithm would be the weighted majority algorithm, which learns the optimal permutation for each $J$. Then $R_i(n) = O(\log n + \log d!) \approx O(\log n + d \log d)$.


%\begin{lemma}
%\label{lem:list regret} Let $\cR$ be any list over $[K]$. Let
%\begin{align}
%  \Delta(\cR) =
%  \sum_{k = 1}^{K - 1} \I{\alpha(\cR(k + 1)) - \alpha(\cR(k)) > 0} (\alpha(\cR(k + 1)) - \alpha(\cR(k)))
%  \label{eq:attraction gap}
%\end{align}
%be the \emph{attraction gap} of list $\cR$. Then the expected regret of $\cR$ is bounded as
%\begin{align*}
%  \sum_{k = 1}^K (\chi(\cR^\ast, k) \alpha(k) - \chi(\cR, k) \alpha(\cR(k))) \leq
%  K \chi_{\max} \Delta(\cR)\,.
%\end{align*}
%\end{lemma}
%\begin{proof}
%Fix position $k \in [K]$. Then
%\begin{align*}
%  \chi(\cR^\ast, k) \alpha(k) - \chi(\cR, k) \alpha(\cR(k))
%  & \leq \chi(\cR^\ast, k) (\alpha(k) - \alpha(\cR(k))) \\
%  & \leq \chi_{\max} (\alpha(k) - \alpha(\cR(k)))\,,
%\end{align*}
%where the first inequality follows from the fact that the examination probability of any position is the lowest in the optimal list (Assumption~\ref{ass:lowest examination}) and the second inequality follows from the definition of $\chi_{\max}$. In the rest of the proof, we bound $\alpha(k) - \alpha(\cR(k))$. We consider three cases.
%
%First, let $\alpha(\cR(k)) \geq \alpha(k)$. Then $\alpha(k) - \alpha(\cR(k)) \leq 0$ and bounded by $\Delta(\cR)$.
%
%Second, let $\alpha(\cR(k)) < \alpha(k)$ and $\pi(k) > k$, where $\pi(k)$ is the position of item $k$ in list $\cR$. Then
%\begin{align*}
%  \alpha(k) - \alpha(\cR(k))
%  & = \alpha(\cR(\pi(k))) - \alpha(\cR(k)) \\
%  & \leq \sum_{i = k}^{\pi(k) - 1} \I{\alpha(\cR(i + 1)) - \alpha(\cR(i)) > 0} (\alpha(\cR(i + 1)) - \alpha(\cR(i)))\,.
%\end{align*}
%From the definition of $\Delta(\cR)$, this quantity is bounded by $\Delta(\cR)$.
% 
%Finally, let $\alpha(\cR(k)) < \alpha(k)$ and $\pi(k) < k$. This implies that there exists an item at a lower position than $k$, $j > k$, such that $\alpha(\cR(j)) \geq \alpha(k)$. Then
%\begin{align*}
%  \alpha(k) - \alpha(\cR(k))
%  & \leq \alpha(\cR(j)) - \alpha(\cR(k)) \\
%  & \leq \sum_{i = k}^{j - 1} \I{\alpha(\cR(i + 1)) - \alpha(\cR(i)) > 0} (\alpha(\cR(i + 1)) - \alpha(\cR(i)))\,.
%\end{align*}
%From the definition of $\Delta(\cR)$, this quantity is bounded by $\Delta(\cR)$. This concludes the proof.
%\end{proof}
%
%\begin{lemma}
%\label{lem:off-base gap} Let
%\begin{align*}
%  \rnd{\cP}_t =
%  \set{(i, j) \in [K]^2: i < j, \ \abs{\bar{\rnd{\cR}}_t^{-1}(i) - \bar{\rnd{\cR}}_t^{-1}(j)} = 1, \
%  \rnd{s}_{t - 1}(i, j) \leq 2 \sqrt{\rnd{n}_{t - 1}(i, j) \log(1 / \delta)}}
%\end{align*}
%be the set of potentially randomized item pairs at time $t$ and $\rnd{\Delta}_t = \max_{\rnd{\cR}_t} \Delta(\rnd{\cR}_t)$ be the \emph{maximum attraction gap} of any list $\rnd{\cR}_t$, where $\Delta(\rnd{\cR}_t)$ is defined in \eqref{eq:attraction gap}. Then on event $\cE$ in \cref{lem:concentration},
%\begin{align*}
%  \rnd{\Delta}_t \leq
%  3 \sum_{i = 1}^K \sum_{j = i + 1}^K \I{(i, j) \in \rnd{\cP}_t} (\alpha(i) - \alpha(j))
%\end{align*}
%holds at any time $t \in [n]$.
%\end{lemma}
%\begin{proof}
%Fix list $\rnd{\cR}_t$ and position $k \in [K - 1]$. Let $i', i, j, j'$ be items at positions $k - 1, \dots, k + 2$ in $\bar{\rnd{\cR}}_t$. If $k = 1$, let $i' = i$; and if $k = K - 1$, let $j' = j$. We consider two cases.
%
%First, suppose that the permutation at time $t$ is such that $i$ and $j$ could be swapped. Then
%\begin{align*}
%  & \alpha(\rnd{\cR}_t^{-1}(k + 1)) - \alpha(\rnd{\cR}_t^{-1}(k)) \leq \\
%  & \quad \I{(\min \set{i, j}, \max \set{i, j}) \in \rnd{\cP}_t} (\alpha(\min \set{i, j}) - \alpha(\max \set{i, j}))
%\end{align*}
%holds on event $\cE$ by the design of $\bubblerank$. More specifically, $(\min \set{i, j}, \max \set{i, j}) \notin \rnd{\cP}_t$ implies that $\alpha(\rnd{\cR}_t^{-1}(k + 1)) - \alpha(\rnd{\cR}_t^{-1}(k)) \leq 0$.
%
%Second, suppose that the permutation at time $t$ is such that $i$ and $i'$ could be swapped, $j$ and $j'$ could be swapped, or both. Then
%\begin{align*}
%  & \alpha(\rnd{\cR}_t^{-1}(k + 1)) - \alpha(\rnd{\cR}_t^{-1}(k)) \leq \\
%  & \quad \I{(\min \set{i, i'}, \max \set{i, i'}) \in \rnd{\cP}_t} (\alpha(\min \set{i, i'}) - \alpha(\max \set{i, i'})) + {} \\
%  & \quad \alpha(j) - \alpha(i) + {} \\
%  & \quad \I{(\min \set{j, j'}, \max \set{j, j'}) \in \rnd{\cP}_t} (\alpha(\min \set{j, j'}) - \alpha(\max \set{j, j'}))
%\end{align*}
%holds by the same argument as in the first case. Also note that
%\begin{align*}
%  \alpha(j) - \alpha(i) \leq
%  \I{(\min \set{i, j}, \max \set{i, j}) \in \rnd{\cP}_t} (\alpha(\min \set{i, j}) - \alpha(\max \set{i, j}))
%\end{align*}
%holds on event $\cE$ by the design of $\bubblerank$.
%
%Therefore, for any position $k \in [K - 1]$ in both above cases,
%\begin{align*}
%  & \alpha(\rnd{\cR}_t^{-1}(k + 1)) - \alpha(\rnd{\cR}_t^{-1}(k)) \leq \\
%  & \quad \sum_{\ell = k - 1}^{k + 1}
%  \I{\left(\min \set{\bar{\rnd{\cR}}_t^{-1}(\ell), \bar{\rnd{\cR}}_t^{-1}(\ell + 1)},
%  \max \set{\bar{\rnd{\cR}}_t^{-1}(\ell), \bar{\rnd{\cR}}_t^{-1}(\ell + 1)}\right) \in \rnd{\cP}_t} \times {} \\
%  & \quad \hspace{0.35in} \left(\alpha\left(\min \set{\bar{\rnd{\cR}}_t^{-1}(\ell), \bar{\rnd{\cR}}_t^{-1}(\ell + 1)}\right) -
%  \alpha\left(\max \set{\bar{\rnd{\cR}}_t^{-1}(\ell), \bar{\rnd{\cR}}_t^{-1}(\ell + 1)}\right)\right)\,.
%\end{align*}
%Now we sum over all positions and note that each pair of $\bar{\rnd{\cR}}_t^{-1}(\ell)$ and $\bar{\rnd{\cR}}_t^{-1}(\ell + 1)$ appears on the right-hand side at most three times, for any list $\rnd{\cR}_t$. This concludes our proof.
%\end{proof}
%
%\begin{lemma}
%\label{lem:off-base regret} Let $\rnd{\cP}_t$ be defined as in \cref{lem:off-base gap}. Then on event $\cE$ in \cref{lem:concentration},
%\begin{align*}
%  \sum_{k = 1}^K (\chi(\cR^\ast, k) \alpha(k) - \chi(\rnd{\cR}_t, k) \alpha(\rnd{\cR}_t(k)))
%  \leq 3 K \chi_{\max} \sum_{i = 1}^K \sum_{j = i + 1}^K \I{(i, j) \in \rnd{\cP}_t} (\alpha(i) - \alpha(j))
%\end{align*}
%holds at any time $t \in [n]$.
%\end{lemma}
%\begin{proof}
%The claim follows directly from chaining \cref{lem:list regret,lem:off-base gap}.
%\end{proof}
%
%\begin{lemma}
%\label{lem:swap bound} Let $\rnd{\cP}_t$ be defined as in \cref{lem:off-base gap}, $\rnd{\cP} = \bigcup_{t = 1}^n \rnd{\cP}_t$, and $\cV_0$ be defined as in \eqref{eq:incorrectly-ordered pairs}. Then on event $\cE$ in \cref{lem:concentration},
%\begin{align*}
%  \abs{\rnd{\cP}} \leq K - 1 + 2 \abs{\cV_0}\,.
%\end{align*}
%\end{lemma}
%\begin{proof}
%From the design of $\bubblerank$, $\abs{\rnd{\cP}_1} = K - 1$. The set of randomized item pairs grows only if the base list in $\bubblerank$ changes. When this happens, the number of incorrectly-ordered item pairs decreases by one, on event $\cE$, and the set of randomized item pairs increases by at most two pairs. This event occurs at most $\abs{\cV_0}$ times. This concludes our proof.
%\end{proof}
%
%\begin{lemma}
%\label{lem:cumulative click loss} For any items $i$ and $j$ such that $i < j$, $\displaystyle \rnd{s}_n(i, j) \leq 15 \frac{\alpha(i) + \alpha(j)}{\alpha(i) - \alpha(j)} \log(1 / \delta)$ on event $\cE$ in \cref{lem:concentration}.
%\end{lemma}
%\begin{proof}
%Let $\rnd{s}_t = \rnd{s}_t(i, j)$ and $\rnd{n}_t = \rnd{n}_t(i, j)$ for any $t \in [n]$.
%
%The proof has two parts. First, suppose that $\rnd{s}_t \leq 2 \sqrt{\rnd{n}_t \log(1 / \delta)}$ holds at all times $t \in [n]$. Then from this assumption and on event $\cE$ in \cref{lem:concentration},
%\begin{align*}
%  \frac{\alpha(i) - \alpha(j)}{\alpha(i) + \alpha(j)} \rnd{n}_t - 2 \sqrt{\rnd{n}_t \log(1 / \delta)} \leq
%  \rnd{s}_t \leq
%  2 \sqrt{\rnd{n}_t \log(1 / \delta)}\,.
%\end{align*}
%This implies that
%\begin{align*}
%  \rnd{n}_t \leq
%  \left[4 \frac{\alpha(i) + \alpha(j)}{\alpha(i) - \alpha(j)}\right]^2 \log(1 / \delta)
%\end{align*}
%at any time $t$, and in turn that
%\begin{align*}
%  \rnd{s}_t \leq
%  2 \sqrt{\rnd{n}_t \log(1 / \delta)} \leq
%  8 \frac{\alpha(i) + \alpha(j)}{\alpha(i) - \alpha(j)} \log(1 / \delta)
%\end{align*}
%at any time $t$. Our claim follows from setting $t = n$.
%
%On the other hand, suppose that $\rnd{s}_t \leq 2 \sqrt{\rnd{n}_t \log(1 / \delta)}$ does not hold at all times $t \in [n]$. Let $\tau$ be the first time when $\rnd{s}_\tau > 2 \sqrt{\rnd{n}_\tau \log(1 / \delta)}$. Then from the definition of $\tau$ and on event $\cE$ in \cref{lem:concentration},
%\begin{align*}
%  \frac{\alpha(i) - \alpha(j)}{\alpha(i) + \alpha(j)} \rnd{n}_\tau - 2 \sqrt{\rnd{n}_\tau \log(1 / \delta)} \leq
%  \rnd{s}_\tau \leq
%  \rnd{s}_{\tau - 1} + 1 \leq
%  2 \sqrt{\rnd{n}_\tau \log(1 / \delta)} + 1 \leq
%  3 \sqrt{\rnd{n}_\tau \log(1 / \delta)}\,,
%\end{align*}
%where the last inequality holds for any $\delta \leq 1 / e$. This implies that
%\begin{align*}
%  \rnd{n}_\tau \leq
%  \left[5 \frac{\alpha(i) + \alpha(j)}{\alpha(i) - \alpha(j)}\right]^2 \log(1 / \delta)\,,
%\end{align*}
%and in turn that
%\begin{align*}
%  \rnd{s}_\tau \leq
%  3 \sqrt{\rnd{n}_\tau \log(1 / \delta)} \leq
%  15 \frac{\alpha(i) + \alpha(j)}{\alpha(i) - \alpha(j)} \log(1 / \delta)\,.
%\end{align*}
%Now note that $\rnd{s}_t = \rnd{s}_\tau$ for any $t > \tau$, from the design of $\bubblerank$. This concludes our proof.
%\end{proof}
%
%For some $\cF_t = \sigma(\rnd{\cR}_1, \rnd{c}_1, \dots, \rnd{\cR}_t, \rnd{c}_t)$-measurable event $A$, let $\mathbb{P}_t(A) = \mathbb{P}(A \mid \cF_t)$ be the conditional probability of $A$ given history $\rnd{\cR}_1, \rnd{c}_1, \dots, \rnd{\cR}_t, \rnd{c}_t$. Let the corresponding conditional expectation operator be $\Et{\cdot}$. Note that $\bar{\rnd{\cR}}_t$ is $\cF_{t - 1}$-measurable.
%
%\begin{lemma}
%\label{lem:click loss} Let $i, j \in [K]$ be any items at consecutive positions in $\bar{\rnd{\cR}}_t$ and
%\begin{align*}
%  \rnd{z} = \rnd{c}_t(\rnd{\cR}_t^{-1}(i)) - \rnd{c}_t(\rnd{\cR}_t^{-1}(j))\,.
%\end{align*}
%Then, on the event that $i$ and $j$ are subject to randomization at time $t$, %\todot{conditional expectation undefined if $\rnd{z} \neq 0$ with prob zero}
%\begin{align*}
%  \Etp{\rnd{z} \mid \rnd{z} \neq 0} \geq
%  \frac{\alpha(i) - \alpha(j)}{\alpha(i) + \alpha(j)}
%\end{align*}
%when $\alpha(i) > \alpha(j)$, and $\Etp{- \rnd{z} \mid \rnd{z} \neq 0} \leq 0$ when $\alpha(i) < \alpha(j)$.
%\end{lemma}
%\begin{proof}
%The first claim is proved as follows. From the definition of expectation and $\rnd{z} \in \set{-1, 0, 1}$,
%\begin{align*}
%  \Etp{\rnd{z} \mid \rnd{z} \neq 0}
%  & = \mathbb{P}_{t - 1}(\rnd{z} = 1 \mid \rnd{z} \neq 0) - \mathbb{P}_{t - 1}(\rnd{z} = -1 \mid \rnd{z} \neq 0) \\
%  & = \frac{\mathbb{P}_{t - 1}(\rnd{z} = 1, \rnd{z} \neq 0) - \mathbb{P}_{t - 1}(\rnd{z} = -1, \rnd{z} \neq 0)}
%  {\mathbb{P}_{t - 1}(\rnd{z} \neq 0)} \\
%  & = \frac{\mathbb{P}_{t - 1}(\rnd{z} = 1) - \mathbb{P}_{t - 1}(\rnd{z} = -1)}{\mathbb{P}_{t - 1}(\rnd{z} \neq 0)} \\
%  & = \frac{\Etp{\rnd{z}}}{\mathbb{P}_{t - 1}(\rnd{z} \neq 0)}\,,
%\end{align*}
%where the third equality follows from $\rnd{z} = 1 \implies \rnd{z} \neq 0$ and $\rnd{z} = -1 \implies \rnd{z} \neq 0$.
%
%Let $\chi_i = \Etp{\chi(\rnd{\cR}_t, \rnd{\cR}_t^{-1}(i))}$ and $\chi_j = \Etp{\chi(\rnd{\cR}_t, \rnd{\cR}_t^{-1}(j))}$ denote the average examination probabilities of the positions with items $i$ and $j$, respectively, in $\rnd{\cR}_t$; and consider the event that $i$ and $j$ are subject to randomization at time $t$. By Assumption~\ref{ass:order-free examination}, the values of $\chi_i$ and $\chi_j$ do not depend on the randomization of other parts of $\bar{\rnd{\cR}}_t$, only on the positions where $i$ and $j$ are. Then $\chi_i \geq \chi_j$; from $\alpha(i) > \alpha(j)$ and Assumption~\ref{ass:examination scaling}. Based on this fact, $\Etp{\rnd{z}}$ is bounded from below as
%\begin{align*}
%  \Etp{\rnd{z}} =
%  \chi_i \alpha(i) - \chi_j \alpha(j) \geq
%  \chi_i (\alpha(i) - \alpha(j))\,,
%\end{align*}
%where the inequality is from $\chi_i \geq \chi_j$. Moreover, $\mathbb{P}_{t - 1}(\rnd{z} \neq 0)$ is bounded from above as
%\begin{align*}
%  \mathbb{P}_{t - 1}(\rnd{z} \neq 0) =
%  \mathbb{P}_{t - 1}(\rnd{z} = 1) + \mathbb{P}_{t - 1}(\rnd{z} = -1) \leq
%  \chi_i \alpha(i) + \chi_j \alpha(j) \leq
%  \chi_i (\alpha(i) + \alpha(j))\,,
%\end{align*}
%where the first inequality is from $\mathbb{P}_{t - 1}(\rnd{z} = 1) \leq \chi_i \alpha(i)$ and $\mathbb{P}_{t - 1}(\rnd{z} = -1) \leq \chi_j \alpha(j)$, and the second inequality is from $\chi_i \geq \chi_j$.
%
%Finally, we chain all above inequalities and get our first claim. The second claim follows from the observation that $\Etp{- \rnd{z} \mid \rnd{z} \neq 0} = - \Etp{\rnd{z} \mid \rnd{z} \neq 0}$.
%\end{proof}
%
%\begin{lemma}
%\label{lem:concentration} Let
%\begin{align*}
%  S_1 = \set{(i, j) \in [K]^2: i < j}\,, \quad
%  S_2 = \set{(i, j) \in [K]^2: i > j}\,. \quad
%\end{align*}
%Let
%\begin{align*}
%  \cE_{t, 1} & = \set{\forall (i, j) \in S_1:
%  \frac{\alpha(i) - \alpha(j)}{\alpha(i) + \alpha(j)} \rnd{n}_t(i, j) - 2 \sqrt{\rnd{n}_t(i, j) \log(1 / \delta)} \leq \rnd{s}_t(i, j)}\,, \\
%  \cE_{t, 2} & = \set{\forall (i, j) \in S_2:
%  \rnd{s}_t(i, j) \leq 2 \sqrt{\rnd{n}_t(i, j) \log(1 / \delta)}}\,.
%\end{align*}
%Let $\cE = \bigcap_{t \in [n]} (\cE_{t, 1} \cap \cE_{t, 2})$ and $\ccE$ be the complement of $\cE$. Then $\mathbb{P}(\ccE) \leq \delta^\frac{1}{2} K^2 n$.
%\end{lemma}
%\begin{proof}
%First, we bound $\mathbb{P}(\overline{\cE_{t, 1}})$. Fix $(i, j) \in S_1$, $t \in [n]$, and $(\rnd{n}_\ell(i, j))_{\ell = 1}^t$. Let
%\begin{align*}
%  \tau(m) = \min \set{\ell \in [t]: \rnd{n}_\ell(i, j) = m}
%\end{align*}
%for $m \in [\rnd{n}_t(i, j)]$. In plain English, $\tau(m)$ is the time of observing item pair $(i, j)$ for the $m$-th time. Let $\rnd{z}_\ell = \rnd{c}_\ell(\rnd{\cR}_\ell^{-1}(i)) - \rnd{c}_\ell(\rnd{\cR}_\ell^{-1}(j))$. Since $(\rnd{n}_\ell(i, j))_{\ell = 1}^t$ is fixed, note that $\rnd{z}_\ell \neq 0$ if $\ell = \tau(m)$ for some $m \in [\rnd{n}_t(i, j)]$. Let
%\begin{align*}
%  \rnd{X}_\ell =
%  \sum_{\ell' = 1}^\ell \condEsub{\rnd{z}_{\tau(\ell')}}{\rnd{z}_{\tau(\ell')} \neq 0}{\tau(\ell') - 1} - \rnd{s}_{\tau(\ell)}(i, j)
%\end{align*}
%for $\ell \in [\rnd{n}_t(i, j)]$ and $\rnd{X}_0 = 0$. Then $(\rnd{X}_\ell)_{\ell = 1}^{\rnd{n}_t(i, j)}$ is a martingale, because
%\begin{align*}
%  \rnd{X}_\ell - \rnd{X}_{\ell - 1}
%  & = \condEsub{\rnd{z}_{\tau(\ell)}}{\rnd{z}_{\tau(\ell)} \neq 0}{\tau(\ell) - 1} -
%  (\rnd{s}_{\tau(\ell)}(i, j) - \rnd{s}_{\tau(\ell - 1)}(i, j)) \\
%  & = \condEsub{\rnd{z}_{\tau(\ell)}}{\rnd{z}_{\tau(\ell)} \neq 0}{\tau(\ell) - 1} -
%  \rnd{z}_{\tau(\ell)}\,,
%\end{align*}
%where the last equality follows from the definition of $\rnd{s}_{\tau(\ell)}(i, j) - \rnd{s}_{\tau(\ell - 1)}(i, j)$. Now we apply the Azuma-Hoeffding inequality and get that
%\begin{align*}
%  P\left(\rnd{X}_{\rnd{n}_t(i, j)} - \rnd{X}_0 \geq 2 \sqrt{\rnd{n}_t(i, j) \log(1 / \delta)}\right) \leq \delta^\frac{1}{2}\,.
%\end{align*}
%Moreover, from the definitions of $\rnd{X}_0$ and $\rnd{X}_{\rnd{n}_t(i, j)}$, and by \cref{lem:click loss}, we have that
%\begin{align*}
%  \delta^\frac{1}{2}
%  & \geq P\left(\rnd{X}_{\rnd{n}_t(i, j)} - \rnd{X}_0 \geq 2 \sqrt{\rnd{n}_t(i, j) \log(1 / \delta)}\right) \\
%  & = P\left(\sum_{\ell' = 1}^{\rnd{n}_t(i, j)} \condEsub{\rnd{z}_{\tau(\ell')}}{\rnd{z}_{\tau(\ell')} \neq 0}{\tau(\ell') - 1} -
%  \rnd{s}_t(i, j) \geq 2 \sqrt{\rnd{n}_t(i, j) \log(1 / \delta)}\right) \\
%  & \geq P\left(\frac{\alpha(i) - \alpha(j)}{\alpha(i) + \alpha(j)} \rnd{n}_t(i, j) -
%  \rnd{s}_t(i, j) \geq 2 \sqrt{\rnd{n}_t(i, j) \log(1 / \delta)}\right) \\
%  & = P\left(\frac{\alpha(i) - \alpha(j)}{\alpha(i) + \alpha(j)} \rnd{n}_t(i, j) -
%  2 \sqrt{\rnd{n}_t(i, j) \log(1 / \delta)} \geq \rnd{s}_t(i, j)\right)\,.
%\end{align*}
%The above inequality holds for any $(\rnd{n}_\ell(i, j))_{\ell = 1}^t$, and thus in expectation over $(\rnd{n}_\ell(i, j))_{\ell = 1}^t$. From the definition of $\cE_{t, 1}$ and the union bound, we have that
%\begin{align*}
%  \mathbb{P}(\overline{\cE_{t, 1}}) \leq
%  \frac{1}{2} \delta^\frac{1}{2} K (K - 1)\,.
%\end{align*}
%The claim that
%\begin{align*}
%  \mathbb{P}(\overline{\cE_{t, 2}}) \leq
%  \frac{1}{2} \delta^\frac{1}{2} K (K - 1)\,.
%\end{align*}
%is proved similarly, except that we use $\condEsub{\rnd{z}_{\tau(\ell)}}{\rnd{z}_{\tau(\ell)} \neq 0}{\tau(\ell) - 1} \leq 0$. From the definition of $\ccE$ and the union bound,
%\begin{align*}
%  \mathbb{P}(\ccE) \leq
%  \sum_{t = 1}^n \mathbb{P}(\overline{\cE_{t, 1}}) + \sum_{t = 1}^n \mathbb{P}(\overline{\cE_{t, 2}}) \leq
%  \delta^\frac{1}{2} K^2 n\,.
%\end{align*}
%This completes our proof.
%\end{proof}
