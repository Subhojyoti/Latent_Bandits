We define $[n] = \lbrace 1,2,\ldots, n\rbrace$ and for any two sets $A$ and $B$, $A^B$ denotes the set of all vectors who take values from $A$ and are indexed by $B$. Let, $M\in [0,1]^{K\times L}$ denote any matrix, then $M(I,:)$ denote any submatrix of $k$ rows such that $I\in[K]^k$ and similarly $M(:,J)$ denote any submatrix of $j$ columns such that $J\in[L]^{j}$.
	
	Let $M$ be reward matrix of  dimension $K\times L$ where $K$ is the number of user or rows and $L$ is the number of arms or columns. Also, let us assume that this matrix  $M$ has a low rank structure of rank $d << \min\lbrace L,K\rbrace$. Let $U$ and $V$ denote the latent matrices for the users and items, which are not visible to the learner such that,
\begin{align*}
	M = UV^{\intercal} \textbf{ \hspace*{4mm}   s.t.   \hspace*{4mm}} U\in [ \mathbb{R}^+ ]^{K\times d} \textbf{, } V\in  [0,1]^{L\times d}.
\end{align*}	  
	
	Furthermore, we put a constraint on $V$ such that, $\forall j\in [L]$, $ \norm{V(j,:)}_1 \leq 1$. 
	
	
\begin{assumption}\textbf{(Hott-Topics)}
\label{assm:hott-topics}
We assume that there exists $d$-column base factors, denoted by $V(J^*,:)$, such that all rows of $V$ can be written as a convex combination of $V(J^*,:)$ and the zero vector and $J^* = [d]$. We denote the column factors by $V^* = V(J*,:)$. Therefore, for any $i\in [L]$, it can be represented by
\begin{align*}
V(i,:) = a_i V(J^*,:) , 
\end{align*}
where $\exists a_i\in [0,1]^{d}$ and $ \norm{a_i}_1 \leq 1$.
\end{assumption}

%In this paper, in addition to the noisy setting explained in section \ref{intro} we first analyze the proposed algorithm in the easier noise free setting. In the noise free setting, the nature reveals the row $i_t$, and when the learner selects the column $j_t$, it observes the mean of the distribution $\bar{R}(i_t,j_t)$.

%\begin{assumption}
%\label{assm:round-robin}
%We assume that nature is revealing the user $i$ in $\bar{R}(i,:), \forall i\in [K]$  in a Round-Robin fashion such that at timestep $t$, nature reveals $i_t = (t \mod K) + 1$.
%\end{assumption}

\begin{assumption}\textbf{(Click Model)}
\label{assm:click-model}
For each user $i_t$ revealed by the nature at round $t$, the learner is allowed to suggest atmost $d$-items, where $d$ is the rank of the matrix $M$. The user can click one, or all, or none of the recommendations. 
\end{assumption}

\begin{discussion}
The above Assumption \ref{assm:click-model} is an instance of the \textit{Document Click Model (DCM)} \citep{craswell2008experimental}. In DCM every user-item pair has a single parameter called the user-item attraction factor which determines the click probability of the user when the item is shown. Since, this click model depends on learning only one attraction factor and is not dependent on the position of the item (PBM) or the decreasing interest of the user (CBM), it often leads to overfitting of model parameters. 
\end{discussion}

%Assumption \ref{assm:click-model} can be conceptualized in the real-world scenario where the learner has  to suggest movies to users and each movie belongs to a different genre (say thriller, romance, comedy, etc). So, the learner can suggest $d$ movies belonging to different genres to each user in a webpage, and the user can click one, or all, or none of the recommended movies (query abandonement).

\textbf{Noise Model:} Our noise model is quite different from the existing stochastic noise model assumptions of various click models. The usual i.i.d Bernoulli reward assumption on the entries of the user-item preference matrix $M$ is not feasible because the hott-topics assumption is required for every realization of the matrix $M$. Hence,  at every timestep $t$, we generate a noisy matrix $\tilde{M}_t = UD_t V^{\intercal}$, where $D_t$ is a diagonal matrix such that $D_t(i,i)\in[0,1]$. Thus, for every such realization of $\tilde{M}_t, \forall t\in [n]$ the hott-topics structure of $M$ is preserved.

The main goal of the learning agent is to minimize the cumulative regret until the end of horizon $n$. We define the cumulative regret, denoted by $\mathcal{R}_n$ as,

\begin{align*}
\mathcal{R}_n = \sum_{t=1}^{n}\bigg\lbrace \sum_{z=1}^{d} \bigg( r_{t}\left(i_{t}, j^* \right) - r_{t}\left( i_{t}, j_{t,z}\right)\bigg)\bigg\rbrace
\end{align*}

where, $j^* = \argmax_{j\in [L]}\lbrace M(i_t,j)\rbrace$ and $j_{t,z}$ be the suggestion of the learner for the $i_t$ -th user for  $z=1,2,\ldots, d$. Note that $r_{t}\left(i_t, j^* \right)\sim \tilde{M}_t\left(i_t, j^*\right)$ and $r_{t}\left(i_t, j_{t,z} \right)\sim \tilde{M}_t\left(i_t, j_{t,z} \right)$. Taking expectation over both sides, we can show that,

\begin{align*}
\E[\mathcal{R}_n] & = \E\left [ \sum_{t=1}^{n}\bigg\lbrace\sum_{z=1}^{d} \bigg( r_{t}\left(i_t, j^* \right) - r_{t}\left( i_t, j_{t,z}\right)\bigg)\bigg\rbrace\right] \\
%%%%%%%%%%%%%%%%%%%%
&= \E\left [ \sum_{t=1}^{n} \sum_{z=1}^{d} \bigg( N_{i_t,j_{z,t}}(t)\bigg) \right ]\Delta_{i_t,j_{t,z}}
\end{align*}

where, $\Delta_{i_t,j_{z,t}} = M(i_t,j^*) - M(i_t,j_{z,t})$ and $N_{i_t,j_{t,z}}(t)$ is the number of times the learner has observed the $j_{t,z}$-th item for the $i_t$-th user. Let, $\Delta = \min_{i\in[K],j\in[L]}\lbrace \Delta_{i,j}\rbrace$ be the minimum gap over all the user, item pair in $M$.
