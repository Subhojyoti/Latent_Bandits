In this paper, we study the problem of recommending the best items to users who are coming sequentially. The learner has access to very less prior information about the users (cold start) and it has to adapt quickly to the user preferences and suggest the best item to each user. Furthermore, we consider a latent variable model such that the user-item preferences are generated as a mixture of user features and item features. Note, that only noisy observations from the user-item preference matrix is only visible to the learner and not the latent user or item features. This results in the user-item preference matrix of being low-rank in nature which is very common in recommender systems \citep{koren2009matrix}, \citep{ricci2011liorrokach}. Also, we assume that each user has a single best item preference.


%Furthermore, we consider the setting where users are grouped into clusters and within each cluster the users have the same choice of the best item, even though their quality of preference may be different for the best item. These clusters along with the choice of the best item for each user are unknown to the learner.  Also, we assume that each user has a single best item preference.

	This complex problem can be conceptualized as a low rank online learning  problem where there are $K$ users and $L$ items. The reward matrix, denoted by $M\in [0,1]^{K\times L}$,  generating the rewards for user, item pair has a low rank structure. The online learning game proceeds as follows, at every timestep $t$,  nature reveals one user (or row) from $M$ where user is denoted by $i_t$. The learner selects some items (or columns) from $M$, where an item is denoted by $j_t\in [L]$. Then the learner receives feedback $r_{t}(i_t,j_t)$ from one noisy realization of $(M(i_t,j_t))$, and $\E[r_{t}(i_t,j_t)] = M(i_t,j_t)$. Then the goal of the learner is to minimize the cumulative regret by quickly identifying the best item $j^*$ for each $i\in [K]$ where $M(i,j^*) = \argmax _{j\in[L]} M(i,j)$. 
	
% where $Ber$ is a Binomial distribution over the entries in $M$