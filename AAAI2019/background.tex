%!TEX root = paper.tex

\newcommand{\transpose}{^\mathsf{\scriptscriptstyle T}}

Let $[n] = \{1, \dots, n\}$ be the set of the first $n$ positive integers. For any two sets $A$ and $B$, we denote by $A^B$ the set of all vectors whose entries take values from $A$ and are indexed by $B$. Let $M$ be a $K \times L$ matrix each of whose entry is in $[0,1]$. For $I \in  [K]^d$, $M(I, :)$ is the $d \times L$ submatrix of $M$ whose $i$th row is same as $I(i)$th row of $M$, for all $ i \in [d]$. Similarly, for $J \in [L]^d,$ $M(:, J)$ corresponds to the $K \times d$ submatrix of $M$ whose columns are appropriate columns of  $M$.  Let $\Pi_d$ be the set of all $d$-permutations. For any $\pi \in \Pi_d$ and $d$-dimensional vector $v$, we denote by $\pi(v)$ the permutation of the entries of $v$ according to $\pi$. \todob{We should say that we often treat sets of row and column indices as vectors.} \todoan{I changed the references to sets in the above to vectors. }

We model user preferences by a family of low-rank matrices, which are known as hott topics. We define a \emph{hott-topics matrix} of rank $d$ as $M = U V\transpose$, where $U$ is a $K \times d$ non-negative matrix and $V$ is a $L \times d$ non-negative matrix that has the hott-topics structure. More precisely, we assume that there exist $d$ rows $J^\ast$ in $V$ such that every row of $V$ can be written as a convex combination of rows $J^\ast$ and the zero vector. More precisely,
\begin{align}
  \forall j \in [L] \ \exists \alpha \in A: V(J^\ast, :) \alpha = V(j, :)\,,
  \label{eq:hott topics}
\end{align}
where $A = \{a \in [0, 1]^{d \times 1}: \|a\|_1 \leq 1\}$.

In the context of user modeling, our factorization of $M$ can be interpreted as follows. The rank $d$ is the number of latent topics. The matrix $U$ are latent preferences of $K$ users over $d$ topics, where $U(i, :)$ are the preferences of user $i \in [K]$. Without loss of generality, we assume that $U \in [0, 1]^{L \times d}$. The matrix $V$ are latent preferences of $L$ items in the space of $d$ topics, where $V(j, :)$ are the coordinates of item $j \in [L]$. Without loss of generality, we assume that the coordinates are points in a simplex, $\|V(j, :)\|_1 \leq 1$ for all $j \in [L]$.
\todoan{Strictly speaking, this is not w.l.o.g - We have $UV \in [0,1]^{K \times L}.$ W.l.o.g, we can either  assume $U \in [0, 1]^{L \times d}$  or we can assume $\|V(j, :)\|_1 \leq 1$.}