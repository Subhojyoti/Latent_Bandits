%!TEX root = paper.tex

\clearpage
\onecolumn
\appendix

\section{Proof}
\label{sec:proof}

The reward for recommending $d$ columns $J$ to user $i$ is
\begin{align*}
  r_t(i, J) =
  \max \, \{\mu(k) \, r_t(i, J(k)): k \in [d]\}
\end{align*}
for weights $\mu(1) \geq \dots \geq \mu(d) > 0$. We also define the corresponding unweighted reward as
\begin{align*}
  \tilde{r}_t(i, J) =
  \max \, \{r_t(i, J(k)): k \in [\text{length}(J)]\}\,.
\end{align*}
Let $J_\ast$ be the indices of hott topics and $\pi_{\ast, i}$ be their highest-reward permutation for user $i$. Let $J_t$ be our recommended columns at time $t$ and $\pi_{t, i}$ be their permutation for user $i$, which is computed by some later-defined row algorithm. Then the expected $n$-step regret of our learning agent is
\begin{align*}
  R(n) =
  \sum_{t = 1}^n \E\left[r_t(i_t, \pi_{\ast, i_t}(J_\ast)) - r_t(i_t, \pi_{t, i_t}(J_t))\right]\,.
\end{align*}
To decompose the regret into its row and column components, we introduce event $1\{J_t \neq J_\ast\}$,
\begin{align*}
  R(n)
  & = \sum_{t = 1}^n \E\left[1\{J_t \neq J_\ast\} (r_t(i_t, \pi_{\ast, i_t}(J_\ast)) - r_t(i_t, \pi_{t, i_t}(J_t)))\right] +
  \sum_{t = 1}^n \E\left[1\{J_t = J_\ast\} (r_t(i_t, \pi_{\ast, i_t}(J_\ast)) - r_t(i_t, \pi_{t, i_t}(J_\ast)))\right] \\
  & \leq \sum_{t = 1}^n \E\left[1\{J_t \neq J_\ast\}\right] +
  \sum_{i = 1}^K \E\left[\sum_{t = 1}^n 1\{i_t = i, J_t = J_\ast\} (r_t(i, \pi_{\ast, i}(J_\ast)) - r_t(i, \pi_{t, i}(J_\ast)))\right]\,,
\end{align*}
where the inequality follows from the fact that the maximum instantaneous regret is $1$.

The first term in the regret decomposition can be bounded as follows. Let
\begin{align*}
  C_{t, i}(J, k) =
  \max \, \{r_t(i, J(\ell)): \ell \in [k]\}\,, \quad
  g_{t, i}(J, k) =
  C_{t, i}(J, k) -   C_{t, i}(J, k - 1)\,.
\end{align*}
Let $J_\ast$ be defined greedily as
\begin{align*}
  J_\ast(k) =
  \argmax_{j \in [L]} \sum_{t = 1}^n \E\left[g_{t, i_t}(J_\ast(: k - 1) \oplus j, k)\right]
\end{align*}
for $k \in [d]$. Then from the design of our column learning algorithm,
\begin{align*}
  \max_{j \in [L]} \sum_{t = 1}^n \E\left[g_{t, i_t}(J_t(: k - 1) \oplus j, k)\right] -
  \sum_{t = 1}^n \E\left[g_{t, i_t}(J_t, k)\right] \leq
  \sqrt{L n}
\end{align*}
for any $k \in [d]$. This implies that
\begin{align*}
  \sum_{t = 1}^n \E\left[g_{t, i_t}(J_t, k)\right]
  & \geq \max_{j \in [L]} \sum_{t = 1}^n \E\left[g_{t, i_t}(J_t(: k - 1) \oplus j, k)\right] - \sqrt{L n} \\
  & \geq \frac{1}{d} \sum_{j \in J_\ast} \sum_{t = 1}^n \E\left[g_{t, i_t}(J_t(: k - 1) \oplus j, k)\right] - \sqrt{L n} \\
  & = \frac{1}{d} \sum_{t = 1}^n \E\left[\sum_{j \in J_\ast} g_{t, i_t}(J_t(: k - 1) \oplus j, k)\right] - \sqrt{L n} \\
  & \geq \frac{1}{d} \sum_{t = 1}^n \E\left[C_{t, i_t}(J^\ast, d)\right] - \sqrt{L n}\,.
\end{align*}
%\todob{The last step is false. The rest of the analysis goes through.} 
Now we prove for any $k \in [d]$,
\begin{align*}
  \sum_{t = 1}^n \E\left[C_{t, i_t}(J^\ast, d)\right] - \sum_{t = 1}^n \E\left[C_{t, i_t}(J_t, k)\right] \leq
  \frac{d - k}{d} \sum_{t = 1}^n \E\left[C_{t, i_t}(J^\ast, d)\right] + k \sqrt{L n}\,.
\end{align*}
The above claim clearly holds for $k = 0$. Now for $k > 0$,
\begin{align*}
  \sum_{t = 1}^n \E\left[C_{t, i_t}(J^\ast, d)\right] - \sum_{t = 1}^n \E\left[C_{t, i_t}(J_t, k)\right]
  & = \sum_{t = 1}^n \E\left[C_{t, i_t}(J^\ast, d)\right] - \sum_{t = 1}^n \E\left[C_{t, i_t}(J_t, k - 1)\right] -
  \sum_{t = 1}^n \E\left[g_{t, i_t}(J_t, k)\right] \\
  & \leq \frac{d - k + 1}{d} \sum_{t = 1}^n \E\left[C_{t, i_t}(J^\ast, d)\right] + (k - 1) \sqrt{L n} -
  \sum_{t = 1}^n \E\left[g_{t, i_t}(J_t, k)\right] \\
  & \leq \frac{d - k}{d} \sum_{t = 1}^n \E\left[C_{t, i_t}(J^\ast, d)\right] + k \sqrt{L n}\,,
\end{align*}
where the first inequality is from our induction hypothesis and the last inequality is from our lower bound on $C_{t, i_t}(J_\ast, d)$. For $k = d$, the claim we have that
\begin{align*}
  \sum_{t = 1}^n \E\left[C_{t, i_t}(J^\ast, d)\right] - \sum_{t = 1}^n \E\left[C_{t, i_t}(J_t, k)\right] \leq
  d \sqrt{L n}\,,
\end{align*}
This is what we needed.

Let
\begin{align*}
  \Delta =
  \min_{t \in [n]} \min_{J:\, J \neq J_\ast} \E\left[\tilde{r}_t(i, J^\ast)\right] - \E\left[\tilde{r}_t(i, J)\right]
\end{align*}
be the minimum gap between the optimal and best suboptimal columns, averaged over users. Then by Lemma~\ref{lem:keylem}, we have
$$    \sum_t \E \tilde{r}_t(i_t, J^*) -  \E \tilde{r}_t(i_t, J_t) \leq   O(d\sqrt{nL}).$$ 
Combining with with the definition of $\Delta$, we get
\begin{align*}
  \sum_{t = 1}^n \E\left[1\{J_t \neq J_\ast\}\right] \leq
 O \left(  \frac{d \sqrt{L n}}{\Delta} \right) \,.
\end{align*}
The second term in the regret decomposition can be bounded as
\begin{align*}
  \sum_{i = 1}^K \E\left[\sum_{t = 1}^n 1\{i_t = i, J_t = J_\ast\} (r_t(i, \pi_{\ast, i}(J_\ast)) - r_t(i, \pi_{t, i}(J_\ast)))\right] \leq
  \sum_{i = 1}^K R_i(n)\,,
\end{align*}
where $R_i(n)$ is the expected $n$-step regret of the row algorithm in row $i$, conditioned on the event that the column algorithm chooses $J_\ast$. One suitable row algorithm is the weighted majority algorithm, which learns the optimal permutation for each $J$. Conditioned on $J$, this is a full-information setting with $d!$ arms. Therefore, $R_i(n) = O(\log n + \log d!) = O(\log n + d \log d)$. Now we chain all above inequalities and get that
\begin{align*}
  R(n) = O\left(\frac{d \sqrt{L n}}{\Delta} + K \log n + K d \log d\right)\,.
\end{align*}

%\todob{Before the we submit the paper, I will add discussion on $\Delta$ and also prove a bound where we integrate it out.}

%For a vector $J$, we will use the notation $J(:k)$ to denote the vector $(J(1),...,J(k)).$
%\begin{lemma}
%\label{lem:keylem}
%
%For any $k \in [d]$,
%$$ \sum_t \E \tilde{r}_t(i_t, J_t[:k]) \geq  \E \sum_t \tilde{r}_t(i_t, J^*[:k]) - O(k\sqrt{nL}).$$ 
%
%\end{lemma}
%\begin{proof}
%We will show this by induction. Note that there are $d$ column MABs. The base case when $k=1$ follows because of the guarantees of MAB$_1(n)$.
%We have by definition $J^*(1) = \max_j   \E \sum_t  {r}_t(i_t, j) $ and $J^*(k) = \max_{j} \E \sum_t   \max \{  \tilde{r}_t(i_t, J^*(:k-1)), r_t(i_t,j) \}$ for $1 < k \leq d$.   We will now assume that the result is true for $k-1$ for some $k>1.$
%We have 
%\begin{align}
%& \E  \sum_t \tilde{r}_t(i_t, J_t(:k))   \\
%&\geq \max_{j} \E \sum_t   \max \{ \tilde{r}_t(i_t, J_t(:k-1)), r_t(i_t,j) \}  - O\left(\sqrt{nL}\right) \\
%&\geq \max_{j} \E \sum_t   \max \{  \tilde{r}_t(i_t, J^*(:k-1)), r_t(i_t,j) \}   - O\left( \sqrt{nL} \right) - O\left( (k-1)\sqrt{nL} \right)   \\
%\label{eq:result}
%& =  \E \sum_t  \tilde{r}_t(i_t, J^*(:k))  - O\left( k\sqrt{nL} \right).
%\end{align}
%%\todob{The last term in the first line misses $\max$. But a more important question is why do we need that term at all. The term is constant throughout the derivation.}
%The last inequality follows from the definition of $J^*.$
%The first inequality is from the guarantees of MAB$_k(n)$. \todob{State the regret bound of Exp3 in a lemma.} The second inequality follows from induction hypothesis and Lemma~\ref{lem:keyinequality}. We note that from Equation~\ref{eq:result}, we have
%$$ \max  \sum_t \tilde{r}_t(i_t, J_t(:k))  \geq  \E \sum_t \tilde{r}_t(i_t, J^*(:k))  - O\left( k \sqrt{nL} \right),$$
%which concludes the proof.
%\end{proof}
%
%
%\begin{lemma}
%\label{lem:keyinequality}
%Suppose
% $$ \E \sum_t \tilde{r}_t(i_t, J_t(:k) \geq  \E \sum_t \tilde{r}_t(i_t, J^*(:k))  - C$$
% %\todob{Define $C$. Why not just $C$?}
%  for some $k \in [d-1]$ and let $j \in [L].$ Then,
%$$ \E \sum_t \max \{ \tilde{r}_t(i_t, J_t(:k)), r_t(i_t,j) \} \geq \E \sum_t  \max \{ \tilde{r}_t(i_t, J^*(:k)), r_t(i_t,j) \} - C.$$
%%\myworries{Both Brano and I have confused ourselves over variants of this lemma. Definitely helpful for everyone to verify if the proof is correct.}
%\end{lemma}
%
%\begin{proof}
%%For any $a,b,c \in \mathbb{R}$ we have $\max \{ a, b\} \geq \max \{ a, c \} +  (b - c).$ Using this, we have
%%\begin{align*}
%%  & \E \sum_t \max \{ \tilde{r}_t(i_t,J_t(:k)),r_t(i_t,j) \} \\
%%  & \quad = \E\sum_{t } \max \{ \tilde{r}_t(i_t,J_t(:k)),r_t(i_t,j) \} \\
%%  &  \quad \geq \E \sum_{t } \max \{ \tilde{r}_t(i_t,J_t(:k)),\tilde{r}_t(i_t,J^*(k)), r_t(i_t,j) \} + \tilde{r}_t(i_t,J_t(:k)) - \tilde{r}_t(i_t,J^*(k)) \\
%%   &  \quad \geq \E \sum_{t } \max \{ \tilde{r}_t(i_t,J_t(:k)),\tilde{r}_t(i_t,J^*(k)), r_t(i_t,j) \} +  \E \sum_{t }  \tilde{r}_t(i_t,J_t(:k)) - \tilde{r}_t(i_t,J^*(k)) \\
%%   &  \quad \geq \E \sum_{t } \max \{ \tilde{r}_t(i_t,J^*(k)), r_t(i_t,j) \} - C.
%%  \end{align*}
%  
%   Let $T_1 = \{ t \, | \tilde r_t(i_t,J^*(:k)) < r_t(i_t,j) \}$ and $T_2 = [n] \backslash T_1 .$ We then have 
%\begin{align*}
%  & \E \sum_t \max \{ \tilde{r}_t(i_t,J_t(:k)),r_t(i_t,j) \} \\
%  & \quad = \E\sum_{t \in T_1} \max \{ \tilde{r}_t(i_t,J_t(:k)),r_t(i_t,j) \} + \E\sum_{t \in T_2} \max \{ \tilde{r}_t(i_t,J_t(:k)),r_t(i_t,j) \}  \\
%  & \quad \geq \E \sum_{t \in T_1} \max \{ \tilde{r}_t(i_t,J^*(:k)),r_t(i_t,j) \} + \E\sum_{t \in T_2} \max \{ \tilde{r}_t(i_t,J_t(:k)),r_t(i_t,j) \} \\
%  & \quad \geq \E \sum_{t \in T_1} \max \{ \tilde{r}_t(i_t,J^*(:k)),r_t(i_t,j) \}) + \E\sum_{t \in T_2} \tilde{r}_t(i_t,J_t(:k)) \\
%  & \quad \geq \E \sum_{t \in T_1}\max \{ \tilde{r}_t(i_t,J^*(:k)),r_t(i_t,j) \} +  \E \sum_{t \in T_2} \tilde{r}_t(i_t,J^*(:k)) - C \\
%  & \quad = \E \sum_{t \in T_1} \max \{ \tilde{r}_t(i_t,J^*(:k)),r_t(i_t,j) \} + \E \sum_{t \in T_2} \max \{ \tilde{r}_t(i_t,J^*(:k)),r_t(i_t,j) \} - C \\
%  & \quad = \E \sum_{t \in [n]}  \max \{ \tilde{r}_t(i_t,J^*(:k)),r_t(i_t,j) \} - C.
%\end{align*}
%%\todob{$(j_1^*,j_2)$ should be $(J^*[1:k-1],j_k)$. $j_2$ should be $j_k$. The same below.}
%%The first inequality is easy because $\max R_t(i_t,(J^*[1:k-1],j) = R_t(i_t,j)$ for $t \in T_1$. Second inequality is trivial. Third inequality follows from the assumption. The next equality holds because of the definition of $T_2$.  
%\end{proof}
%
%
%%\newcommand{\transpose}{^\mathsf{\scriptscriptstyle T}}
%%
%%\section{Proof}
%%\label{sec:proof}
%%
%%Our learning agent operates in the following setting. Let $i_1, \dots, i_n$ be a fixed sequences of users in $n$ steps, which is unknown to the agent. Let $r_t = U D_t V\transpose$ be the reward matrix at time $t$, where $U$ is a non-negative matrix, $D_t$ is a non-negative diagonal matrix that may change with $t$, and $V$ is a non-negative hott-topics matrix. We assume that $r_t \in [0, 1]^{K \times L}$ at all times $t \in [n]$. The only randomness in our problem is due to the learning agent.
%%
%%The reward for recommending $d$ columns $J$ to user $i$ is
%%\begin{align*}
%%  r_t(i, J) =
%%  \max \, \{\mu(k) \, r_t(i, J(k)): k \in [d]\}
%%\end{align*}
%%for weights $\mu(1) \geq \dots \geq \mu(d) > 0$. We also define the corresponding unweighted reward as
%%\begin{align*}
%%  \tilde{r}_t(i, J) =
%%  \max \, \{r_t(i, J(k)): k \in [d]\}\,.
%%\end{align*}
%%Let $J_\ast$ be the indices of hott topics and $\pi_{\ast, i}$ be their highest-reward permutation for user $i$. Let $J_t$ be our recommended columns at time $t$ and $\pi_{t, i}$ be their permutation for user $i$, which is computed by some later-defined row algorithm. The expected $n$-step regret, where the only randomness is due to the learning agent, is
%%\begin{align*}
%%  R(n) =
%%  \E\left[\sum_{t = 1}^n r_t(i_t, \pi_{\ast, i_t}(J_\ast))\right] - \E\left[\sum_{t = 1}^n r_t(i_t, \pi_{t, i_t}(J_t))\right]\,.
%%\end{align*}
%%The regret of the column learning algorithm in $n_0$ steps is bounded as
%%\begin{align*}
%%  \E\left[\sum_{t = 1}^{n_0} \tilde{r}_t(i_t, J_\ast)\right] - \E\left[\sum_{t = 1}^{n_0} \tilde{r}_t(i_t, J_t)\right] \leq
%%  d \sqrt{L n_0}
%%\end{align*}
%%for any $n_0$, based on a similar analysis to ranked bandits. Let
%%\begin{align*}
%%  \Delta = \min_{i \in [K], t \in [n]} \left(\tilde{r}_t(i, J_\ast) - \max_{J:\, J \neq J_\ast} \tilde{r}_t(i, J)\right)
%%\end{align*}
%%be the minimum gap. \todob{The above definition of the gap needs to be adjusted. It is zero whenever $J$ contains the optimal column for user $i$. Our algorithm is sound and learns $J^\ast$. So this is just a technicality.} Then, based on the above inequalities, the probability that the column learning algorithm chooses $J_\ast$ at any time $t \geq n_0$ is bounded from below by
%%\begin{align}
%%  1 - \frac{d \sqrt{L n_0}}{\Delta n_0} =
%%  \frac{\Delta \sqrt{n_0} - d \sqrt{L}}{\Delta \sqrt{n_0}}
%%  \label{eq:opt lower bound}
%%\end{align}
%%for any $\Delta \geq d \sqrt{L / n_0}$.
%%
%%Let $p_t$ be the probability that the column learning algorithm chooses $J_\ast$ at time $t$ and let $\pi_{t, i}(J_\ast)$ be its permutation for user $i$ at time $t$, according to our row algorithm. Then we can bound the regret from time $n_0$ as
%%\begin{align*}
%%  R(n)
%%  & = \sum_{i = 1}^K \E\left[\sum_{t = n_0}^n 1\{i_t = i\}
%%  (r_t(i, \pi_{\ast, i}(J_\ast)) - r_t(i, \pi_{t, i}(J_t))\right] \\
%%  & = \sum_{i = 1}^K \E\left[\sum_{t = n_0}^n \frac{1}{p_t} 1\{i_t = i, J_t = J_\ast\}
%%  (r_t(i, \pi_{\ast, i}(J_\ast)) - r_t(i, \pi_{t, i}(J_\ast))\right] \\
%%  & \leq \left(1 + \frac{d \sqrt{L}}{\Delta \sqrt{n_0} - d \sqrt{L}}\right) \sum_{i = 1}^K R_i(n)\,,
%%\end{align*}
%%where $R_i(n)$ is the expected $n$-step regret of the row algorithm in row $i$, conditioned on the fact that the column learning algorithm chooses $J_\ast$. One suitable row algorithm would be the weighted majority algorithm, which learns the optimal permutation for each $J$. Then $R_i(n) = O(\log n + \log d!) \approx O(\log n + d \log d)$.
%%
%%In the first $n_0$ steps, we bound the regret trivially by $n_0$. Then the expected $n$-step regret is bounded up to log factors as
%%\begin{align*}
%%  R(n) \leq
%%  n_0 + \left(1 + \frac{d \sqrt{L}}{\Delta \sqrt{n_0} - d \sqrt{L}}\right) K\,.
%%\end{align*}
%%The bound can be interpreted as follows. Choose some reasonable $n_0$ that makes the regret comparable to ranked bandits, such as $n_0 = c^2 d \sqrt{L n}$ for some $c > 0$. Then the multiplier at $K$ becomes smaller than $2$, for instance, when $\Delta c d^\frac{1}{2} L^\frac{1}{4} n^\frac{1}{4} \geq 2 d L^\frac{1}{2}$. Under the assumption that $n \geq L$, this happens when $\Delta \geq 2 \sqrt{d} / c$, which makes sense for $c > 2 \sqrt{d}$\,.


\section{Rank 2 case}

\begin{assumption} \textbf{(Distinguishable hott-topics)}
We assume that we have distinguishable hott-topics in $V$ such that for a  $V(J^*,:)$ where rows in $V$ are convex-combination of rows of $J^*$ we have,
\begin{align*}
\|J^*(:,1)\|_1 > \|J^*(:,2)\|_1 > \ldots > \|J^*(:,d)\|_1 .
\end{align*}
\end{assumption}


	\textbf{Noise Model:} At every timestep $t$ we generate the matrix $M_{t} = UD_t V^{\intercal}$, where $D_t$ is a diagonal matrix of dimension $d$.

\begin{assumption} \textbf{(Piecewise i.i.d)} We assume that there exist $g = 1,2,\ldots,G$ changepoints at $t_1 < t_2 < \ldots < t_G$ such that for the each time $t\in [t_{g-1}, t_g]$, $D_t(i,i)\sim \mathcal{X}_g (\mu(D(i,i)),\sigma^2)$. We assume $\mathcal{X}_g (\mu(D(i,i)),\sigma^2)$ is a sub-Gaussian distribution with mean $\mu$ and variance $\sigma^2$.
\end{assumption}

We have to show that timestep $t=1,2,\ldots,n$ the hott-topics structure is maintained even though user preference changes over time in our noisy model. Now we device a worst-case scenario such that the time horizon $n$ is divided into $G$ equal slices, each of length $\dfrac{n}{G}$. Then we show that for each of these time slices the hott-topics structure is maintained with high probability.

%Let $S_t(i,j) = \sum_{i=1}^{d}\sum_{j=1}^{d} U(i,j) D_t(i,j)$ be a random variable. 

Let $S_t = UD_t$. Then we can show that for a $\Delta = \sqrt{\dfrac{L\log(n/G)}{n/G}}$,

\begin{align*}
\sum_{t=1}^{n/G}\Pb\lbrace  S_t(i,j) > U(i,j) \mu(D(i,j)) + \Delta\rbrace &\leq \sum_{t=1}^{n/G}\exp(-2\Delta^2 t)\\
%%%%%%%%%%%%%%%%%%
& \leq \sum_{t=1}^{n/G}\exp(-\dfrac{L\log(n/G)}{n/G}\dfrac{n}{G})\\
%%%%%%%%%%%%%%%%%%
& = \sum_{t=1}^{n/G}\left(\dfrac{G}{n}\right)^L =\left(\dfrac{n}{G}\right)^{1-L}
\end{align*}

Similarly, we can calculate for the other tail, 

\begin{align*}
\sum_{t=1}^{n/G}\Pb\lbrace  S_t(i,j) < U(i,j) \mu(D(i,j)) + \Delta\rbrace &\leq \sum_{t=1}^{n/G}\exp(-2\Delta^2 t) = \left(\dfrac{n}{G}\right)^{1-L}
\end{align*}

So, for every $n/G$ such slices we can show that the hott-topics structure is maintained for $\dfrac{n}{G} - 2\left(\dfrac{n}{G}\right)^{1-L}$ of the timesteps.



%\subsection{Problem Complexity}
%
%Without loss of generality we can assume that $1\geq \mu(1) > \mu(2) \geq 0$. Let $\Delta_{i,j} = \mu(i,1) - \mu(i,j)$. Following \citet{audibert2010best} we define the problem complexity of our setting as 
%
%\begin{align*}
%H_{1} = \sum_{j=1}^{L}\dfrac{1}{\Delta_{i,j}}
%\end{align*} 
%
%
%
%Case 1: $d$ items in 





